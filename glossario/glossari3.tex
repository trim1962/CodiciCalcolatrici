%a
\newglossaryentry{angolog}{name={Angolo},
description={parte di piano compresa fra due semirette che hanno la stessa origine}}
\newglossaryentry{angologirog}{name={Angolo giro},
description={un angolo in cui i due lati coincidono}}
\newglossaryentry{angolopiattg}{name={Angolo piatto},description=
{un angolo che è la metà di un angolo giro}}
\newglossaryentry{angolorettog}{name={Angolo retto},description=
{un angolo che è la metà di un angolo piatto}}
\newglossaryentry{angoloacutog}{name={Angolo acuto},description=
{un angolo che è minore di un angolo retto}}
\newglossaryentry{angolottusog}{name={Angolo ottuso},description=
{un angolo che è maggiore angolo retto}}
\newglossaryentry{angolopositivog}{name={Angolo positivo},description=
{un angolo che per costruirlo si è utilizzato un movimento antiorario}}
\newglossaryentry{angolnegativoog}{name={Angolo negativo},description=
{un angolo che per costruirlo si è utilizzato un movimento orario}}


%C
\newglossaryentry{coorpolog}{name={Coordinate polari},
description={un sistema di coordinate polari individua la posizione di un punto $P$ nel piano, tramite una coppia $(\rho;\theta)$ dove il primo numero è la distanza del punto detto polo e da un angolo $\theta$ misurato in senso antiorario da una semiretta di origine il polo}}
\newglossaryentry{circgoniog}{name={Circonferenza goniometrica},description=
{circonferenza con centro nell'origine e di raggio unitario}}
%D

%E

%F
\newglossaryentry{fasenumcompg}{name={Fase},
description=angolo che nel piano complesso un vettore forma con l'asse reale}
\newglossaryentry{figconveg}{name={Figura convessa},
description={una figura è convessa se presi due qualunque suoi punti il segmento che li congiunge appartiene alla figura}}
\newglossaryentry{figconcag}{name={Figura concava},
description={una figura è convessa se presi due qualunque suoi punti il segmento che li congiunge non appartiene alla figura}}
%G
\newglossaryentry{gradog}{name={Grado sessagesimale},description=
{è la trecentosessantesima parte di un angolo giro}}

%I

%M
\newglossaryentry{minutog}{name={Minuto},description=
{è la sessantesima parte di un grado sessagesimale}}
%N
\newglossaryentry{numcompg}{name={Numero complesso},
description=un numero complesso ${z=a+\uimm b}$}
\newglossaryentry{coniunumcompg}{name={Numero coniugato},
description=il coniugato di un numero complesso è il numero con la parte immaginaria opposta ${z=a+\uimm b}$}
%%P
\newglossaryentry{pianoArganGaussg}{name={Piano di Argand-Gauss},
description=piano complesso in cui l'asse $x$ è l'asse reale mentre l'asse $y$ è la retta immaginaria}

%Q

%R

%S
\newglossaryentry{secondog}{name={Secondo},description=
{è la sessantesima parte di un minuto}}
%U
\newglossaryentry{Uniimgg}{name={Unità immaginaria},
description=simbolo che ha la proprietà che se è elevato al quadrato vale meno uno}

	