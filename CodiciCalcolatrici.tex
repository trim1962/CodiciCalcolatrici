% !BIB TS-program = biber
% !TeX encoding = UTF-8
% !TeX spellcheck = it_IT

\documentclass[openany,a4paper]{book}%
\input{../Mod_base/base}
%\geometry{top=2cm,bottom=2cm,left=2cm,right=2cm}
\usepackage[big]{layaureo}
\input{../Mod_base/grafica}
\input{../Mod_base/matematica}
% !TeX encoding = UTF-8% !TeX encoding = UTF-8
% !TeX spellcheck = it_IT
\documentclass[openany,a4paper]{book}%
% see http://www.tex.ac.uk/cgi-bin/texfaq2html?label=noroom
\usepackage{base}
\usepackage[big]{layaureo}
\usepackage{copyright}
\usepackage{grafica}
\usepackage{stand_class}
\usepackage{matematica}
\usepackage{amsthm}
\usepackage{tabelle}
\DeclareCaptionFormat{grafico}{\textbf{Grafico \thefigure}#2#3}
\DeclareCaptionFormat{esempio}{\textbf{Esempio \thefigure}#2#3}
\newcolumntype{L}{>{$\displaystyle}l<{$}}
\newcolumntype{C}{>{$\displaystyle}c<{$}}   
\newcolumntype{R}{>{$\displaystyle}r<{$}}   
%\newcolumntype{R}{>{$\displaystyle}r<{$}}
%\newcolumntype{T}{>{\centering\arraybackslash}p{1em}} 
%\newcolumntype{W}{>{\sffamily\Large $}c<{$}}
%\newcolumntype{N}[1]{>{\centering\rule[-1mm]{0pt}{4.75mm}}m{#1}}
\newcolumntype{M}[1]{>{\centering}p{#1}}
\newcommand\pilH{\rule{0pt}{2.5ex}}
\newcommand\pilD{\rule[-1ex]{0pt}{0pt}}
\newlength{\gnat}
\newlength{\gnam}
\usepackage{imakeidx}
\makeindex[options=-s ../Mod_base/oldclaudio.sti]
\usepackage{date}
\usepackage{pagina}
\usepackage{unita_misura}
\usepackage[grumpy,mark,markifdirty,raisemark=0.95\paperheight]{gitinfo2}
\usepackage[toc,page]{appendix}

\renewcommand{\appendixtocname}{Appendice}

\renewcommand{\appendixpagename}{Appendice}
\usepackage[italian]{varioref}
\usepackage{hyperxmp}
\usepackage[pdfpagelabels]{hyperref}
\usepackage[italian,noabbrev]{cleveref}
\usepackage{tcolorboxgest}
\title{Esercizi svolti terzo}
\author{Claudio Duchi}
\date{\datetime}
\hypersetup{%
	pdfencoding=auto,
	urlcolor={blue},
	pdftitle={Appunti matematica terzo manutenzione},
	pdfsubject={terzo},
	pdfstartview={FitH},
	pdfpagemode={UseOutlines},
	pdflicenseurl={http://creativecommons.org/licenses/by-nc-nd/3.0/},
	pdflang={it},
	pdfmetalang={it},
	pdfkeywords={goniometria, trigoniometria, numeri complessi},
	pdfcopyright={Copyright (C) 2019, Claudio Duchi},
	pdfcontacturl={http://breviariomatematico.altervista.org},
	pdfcontactpostcode={},
	pdfcontactphone={},
	pdfcontactemail={claduc},
	pdfcontactcountry={Italy},
	pdfcontactcity={Perugia},
	pdfcontactaddress={},
	pdfcaptionwriter={Claudio Duchi},
	pdfauthortitle={},%
	pdfauthor={Claudio Duchi},
	linkcolor={blue},
	colorlinks=true,
	citecolor={red},
	breaklinks,
	bookmarksopen,
	verbose,
	baseurl={http://breviariomatematico.altervista.org}
}


\newcommand{\tabincludegraphics}[2][]{%
	$\vcenter{\hbox{\includegraphics[#1]{#2}}}$}
%\newcommand{\tabincludestandalone}[2][]{%
%	$\vcenter{\hbox{\includestandalone[#1]{#2}}}$}


\newcommand{\HRule}{\rule{\linewidth}{0.5mm}}
\makeatletter
\renewcommand\frontmatter{%
	\cleardoublepage
	\@mainmatterfalse
	%\pagenumbering{roman}
}
\renewcommand\mainmatter{%
	\cleardoublepage
	\@mainmattertrue
	%\pagenumbering{arabic}
}
\makeatother
\usepackage{indice}
\usepackage{utili}

\usepackage{CDloghi}
\listfiles
%secondo/scomposizionipoli,
  secondo/equazioni1,
% secondo/equazioni_fraz,
  %quarto/disequazioni_primogrado,
 secondo/sistemi_lineari,
	secondo/semplificazioni,
	secondo/radicali,
	secondo/Equazioni2grad,
	
%%%%%%%%%%%%%%%%%%%%%%%%%%%%%%%%
%%%lunghezza arrotondamenti%%%%%
\newcommand{\lungarrotandamento}{4}
%%%%%%%%%%%%%%%%%%%%%%%%%%%%%%%%%%
\begin{document}
	\frontmatter
	\begin{titlepage}
		\begin{center}
			\Lgrandedue\\[1cm]
			\textsc{\LARGE Claudio Duchi}\\[1.2cm]
			\HRule \\[0.4cm]
			{ \huge \bfseries APPUNTI DI MATEMATICA}\\[0.4cm]
			{\LARGE \textsc{TERZO MANUTENZIONE}}
			\HRule \\[1.2cm]
			\vfill
			\polylogo[5.5]{15}		
			{\large $-$\DTMnow$-$}
		\end{center}
		{\centering
			Release:\gitReln\ (\gitAbbrevHash)\ Autore:\gitAuthorName\ 
			\gitCommitterDate \\
		}
	\end{titlepage}
	\setcounter{page}{2}
	\CDcopyright
	\tableofcontents 
	\listoftables
	\addcontentsline{toc}{chapter}{\listtablename}
	\listoffigures
	\addcontentsline{toc}{chapter}{\listfigurename}
	\renewcommand\lstlistlistingname{Elenco esempi}
	\addcontentsline{toc}{chapter}{\lstlistlistingname}
	\addcontentsline{toc}{section}{Esempi}
	\lstlistoflistings{}
	{
		%	https://tex.stackexchange.com/questions/318486/number-freestyle-causes-an-overlay-in-the-list-of-tcolorboxes/318512#318512
		\makeatletter
		\renewcommand{\l@tcolorbox}{\@dottedtocline{1}{0pt}{3em}}
		\makeatother
		\tcblistof[\section*]{thm}{Esempi}
		\addcontentsline{toc}{section}{Contro esempi}
		\tcblistof[\section*]{cthm}{Contro esempi}
	}
	%\listoftodos
	\mainmatter%
	\include{scomposizionipoli}

   \chapter{Equazioni}
\label{sec:equazioni}
\section{Definizioni}
\begin{definizionet}{}{}
Un'equazione\index{Equazione} è l'uguaglianza fra due espressioni.
\end{definizionet}
 Dato che dipende dai valori che vengono assegnati alle lettere un'equazione è un'uguaglianza condizionata.
\begin{definizionet}{}{}
  I valori, che sostituiti alle lettere rendono vere l'uguaglianza,  sono chiamate soluzioni\index{Equazione!soluzione}
\end{definizionet}
\begin{esempiot}{}{}
Classificare le seguenti uguaglianze
\end{esempiot}
\begin{itemize}
\item $2+3=3+5$ non è un'equazione. Mancano le lettere.
\item $2+x=3x+5y$ è un'equazione. 
\item $z+3x=0$ è un'equazione.
\end{itemize}
\begin{esempiot}{}{}
Verificare se $x=7$ è soluzione di
 l'equazione $2x-5=x+2$ 
\end{esempiot} 
 $x=7$ è soluzione infatti $2\cdot 7-5=7+2$ $9=9$ l'uguaglianza è verificata. Mentre per $x=3$ $2\cdot 3-5=7+2$ $1\neq5$ quindi non è soluzione.

Le lettere sono chiamate variabili\index{Equazione!variabile} o costanti\index{Equazione!costante}. Una variabile o incognita è una quantità non nota a priori che può assumere qualunque valore. Una costante è una quantità non nota ma fissa. Di solito si usano $x,y,z$ per indicare le incognite $a,b,c,\dots$ per indicare le costanti.  

Un'equazione in cui compare una sola lettera è detta in una incognita, con due diverse in due incognite eccetera. Il segno di uguaglianza divide l'equazione in due parti, la parte a sinistra chiamata primo membro, la parte a destra secondo membro.
\section{Principi di equivalenza}
\begin{definizionet}{}{}
Due o più equazioni sono equivalenti\index{Equazione!equivalente} se hanno le stesse soluzioni.
\end{definizionet}
\begin{esempiot}{}{}
Verificare se le due equazioni sono equivalenti 
\begin{align*}
5x+4&=4x+3\\
3x+2&=2x+3
\end{align*}
\end{esempiot}
le equazioni 
hanno la stessa soluzione
\begin{align*}
5x+4&=4x+3\\
5\cdot(-1)+4&=4\cdot(-1)+3\\
-1&=-1\\
3x+2&=2x+3\\
3\cdot(-1)+2&=2\cdot(-1)+3\\
-1&=-1
\end{align*}
Quindi le due equazioni sono equivalenti.

\subsection{Primo principio di equivalenza}
\label{sec:PrimoprincipioEquivalenza}
\begin{principiot}{Primo principio}{}
Se aggiungiamo o togliamo la stessa quantità\footnote{Quantità definita} al primo e al secondo membro di una equazione,  otteniamo un'equazione  equivalente\index{Equazione!equivalente} a quella di partenza.
\end{principiot}
\begin{esempiot}{Primo principio equivalenza}{}
$8x+14=6x+10$
\end{esempiot}
\begin{align*}
8x+14&=6x+10
\intertext{aggiungendo $+10$ ad entrambi i membri}
8x+14+10&=6x+10+10\\
8x+24&=6x+20
\intertext{le due equazioni sono equivalenti infatti}
8\cdot(-2)+14&=6\cdot(-2)+10\\
8&=8\\
8\cdot(-2)+14&=6\cdot(-2)+10\\
-2&=-2
\intertext{quindi $-2$ è soluzione per entrambe}
\end{align*}
\subsection{Conseguenze primo principio}
Se un termine passa  dal primo al secondo membro di una equazione o viceversa cambia di segno\index{Regola!del trasporto}.
\begin{esempiot}{Conseguenze primo principio}{}
$ x+5 = 8$
\end{esempiot}
\begin{NodesList}
	\tikzset{LabelStyle/.style = {left=0.1cm,pos=0.5,text=red,fill=white}}
 \[ % formula no "inline"
 \begin{aligned}
 x+5 &= 8 \AddNode\\
 x +5-5 &= 8-5 \AddNode\\
 x + 0 &= 8-5 \AddNode
 \end{aligned}
 \]
 \LinkNodes{\begin{minipage}{2cm}
aggiungo $-5$ ad entrambi i membri
 \end{minipage}
 }
 \LinkNodes{ $5-5=0$ }
 \end{NodesList}
 in pratica
 \begin{NodesList}[dy=5pt,margin=3cm]
  \[ % formula no "inline"
  \begin{aligned}
  x+5 &= 8 \AddNode\\
  x  &= 8-5 \AddNode
  \end{aligned}
  \]
  \LinkNodes{\begin{minipage}{2cm}
 sposto e cambio di segno
  \end{minipage}
  }
  \end{NodesList}
Se la stessa quantità è presente nel primo o secondo membro dell'equazione allora può essere eliminata\index{Regola!cancellazione}.
\begin{esempiot}{Regola di cancellazione}{}
$ x+5 = 8+5$
\end{esempiot}
\begin{NodesList}[dy=5pt,margin=3cm]
 \[ % formula no "inline"
 \begin{aligned}
 x+5 &= 8+5 \AddNode\\
 x +5-5 &= 8+5-5 \AddNode\\
 x + 0 &= 8+0 \AddNode
 \end{aligned}
 \]
 \LinkNodes{\begin{minipage}{2cm}
aggiungo $-5$ ad entrambi i membri
 \end{minipage}
 }
 \LinkNodes{ $5-5=0$ }
 \end{NodesList}
 in pratica
 \begin{NodesList}[dy=5pt,margin=3cm]
  \[ % formula no "inline"
  \begin{aligned}
  x+5 &= 8+5 \AddNode\\
  x  &= 8 \AddNode
  \end{aligned}
  \]
  \LinkNodes{\begin{minipage}{2cm}
semplifico
  \end{minipage}
  }
  \end{NodesList}

\subsection{Secondo principio di equivalenza}
\label{sec:SecondoprincipioEquivalenza}
\begin{principiot}{Secondo principio}{}
Se moltiplichiamo o dividiamo per  la stessa quantità diversa da zero il primo e il secondo membro di una equazione,  otteniamo un'equazione  equivalente\index{Equazione!equivalente} a quella di partenza.
\end{principiot}
\begin{esempiot}{Secondo principio di equivalenza}{}
$2x+2=x+5$
\end{esempiot}
\begin{align*}
2x+2&=x+5
\intertext{moltiplico per  $+5$  entrambi i membri}
5\cdot(2x+2)&=5\cdot(x+5)\\
10x+10&=5x+25
\intertext{le due sono equivalenti infatti}
2\cdot(3)+2&=3+5\\
8&=8\\
10\cdot(3)+10&=5\cdot(3)+25\\
40&=40
\intertext{quindi $-2$ è soluzione per entrambe}
\end{align*}
\section{Forma normale}
\label{sec:formanormale}
\begin{definizionet} {}{}
Dopo aver trasportato a primo membro tutti i termini di una equazione si ottiene un polinomio ordinato e l'equazione diventa \[P(x)=0\]
In questo caso l'equazione\index{Equazione!forma normale} si dice in forma normale.
\end{definizionet}
\begin{esempiot}{Forma normale}{}
L'equazione\[3x^2+3=0\] è in forma normale.

L'equazione\[(x-2)(3x+2)+5x=0\] non è in forma normale.
\end{esempiot}
Il grado più grande dell'equazione rispetto all'incognita è detto grado dell'equazione\index{Equazione!grado}.
\section{Equazioni di primo grado}
\label{sec:equazionidiprimogrado}
\begin{definizionet}{}{}
Una equazione\index{Equazione!di primo grado} di primo grado è un'equazione della forma \[ax=b\]
$a,b\in\R$
\end{definizionet}
\begin{esempiot}{}{}
Risolvere l'equazione $8x+2(x+11) = 6x+3(x-3) $
\end{esempiot}
 \begin{NodesList}[margin=3cm]
  \begin{align*}
  8x+2(x+11) = 6x+3(x-3) \AddNode\\
  8x+2x+2  = 6x+3x-9 \AddNode\\
  \intertext{\hfil isolati i termini con l'incognita \hfil}
  8x+2x-6x-3x  = -2-9 \AddNode\\
    x  = -11 \AddNode
  \end{align*}
  \LinkNodes{Moltiplico}
  \LinkNodes{ }
  \LinkNodes{ Ottengo la soluzione}
  \end{NodesList}
  \[x=-11\]
  è soluzione
Nell'esempio che segue l'equazione è scritta in maniera più complessa. Questo impone delle priorità nella risoluzione della stessa. 
\begin{esempiot}{}{}
Risolvere l'equazione \[2(x+\dfrac{4}{3})-\dfrac{5x-3}{2}=2x+3(x+2) \]
\end{esempiot}
 \begin{NodesList}[margin=3cm]
  \begin{align*}
  2\overbrace{(x+\dfrac{4}{3})}-\dfrac{5x-3}{2}=2x+\overbrace{3(x+2)} \AddNode\\
%  \intertext{\hfil isolati i termini con l'incognita \hfil}
\overbrace{2(\dfrac{3x+4}{3})}-\dfrac{5x-3}{2}=\overbrace{2x+3x}+6 \AddNode\\
  \dfrac{6x+8}{3}-\dfrac{5x-3}{2}=5x+6   \AddNode\\
 \dfrac{12x+16-15+9=30x+31}{6}   \AddNode\\
 -3x+25=30x+31\AddNode\\
  \intertext{\hfil isolati i termini con l'incognita \hfil}
 \overbrace{-3x-30x}=\overbrace{-25+31}\AddNode\\
 -33x=6\AddNode\\
 x=-\dfrac{6}{33}\AddNode\\
 x=-\dfrac{2}{11}\AddNode\\
  \end{align*}
  \LinkNodes{Precedenze}
   \LinkNodes{Moltiplico e sommo}
  \LinkNodes{mcm}
  \LinkNodes{Moltiplico per $6$}
  \LinkNodes{Sommo}
  \LinkNodes{Separo}
  \LinkNodes{Divido}
    \LinkNodes{Semplifico}
  \end{NodesList}
Le due equazioni precedenti, in origine, non erano in forma normale\index{Equazione!forma normale}, semplificando e separando le variabili otteniamo un'equazione in forma normale,  che viene risolta dividendo per il termine davanti l'incognita. Dato che abbiamo ottenuto una soluzione l'equazione è determinata\index{Equazione!determinata}.
 
\begin{esempiot}{}{}
Risolvere l'equazione \[2(3x+2)=3(\dfrac{4}{3}x-1)+2(x+1) \]
\end{esempiot}
 \begin{NodesList}[margin=3cm]
  \begin{align*}
\overbrace{2(3x+2)}=3(\dfrac{4}{3}x-1)+\overbrace{2(x+1)} \AddNode\\
%  \intertext{\hfil isolati i termini con l'incognita \hfil}
6x+3=\overbrace{3(\dfrac{4x-3}{3})}+2x+2 \AddNode\\
 6x+3=4x-3+2x+2  \AddNode\\
  \intertext{\hfil isolati i termini con l'incognita \hfil}
 \overbrace{6x-4x-2x}=\overbrace{-4-3+2}  \AddNode\\
 0=-5\AddNode
  \end{align*}
  \LinkNodes{Precedenze}
   \LinkNodes{Moltiplico}
  \LinkNodes{Separo}
  \LinkNodes{Sommo}
  \end{NodesList}
  a  primo membro abbiamo zero al secondo meno cinque. L'uguaglianza è impossibile l'equazione è impossibile.
\begin{figure}
	\centering
	\includestandalone[width=.3\linewidth]{secondo/diagrammi/AlberoBinario1}
	\caption[]{Classificazione equazioni}
	\label{fig:AlberoBinarioeqa1}
\end{figure}

Anche ne caso che segue l'incognita scompare solo che cambia il tipo della soluzione. Nell'esempio che segue scompare l'incognita ma l'uguaglianza ottenuta non è sempre falsa ma sempre vera. L'uguaglianza è un'identità.
\begin{esempiot}{}{}
Risolviamo l'equazione \[ 6(x-3) = 3(x-1)+5(x+\dfrac{2}{5})-(2x+17)\]
\end{esempiot}
\begin{NodesList}[margin=3cm]
  \begin{align*}
\overbrace{6(x-3)} = \overbrace{3(x-1)}+5(\overbrace{x+\dfrac{2}{5}})\overbrace{-(2x+17)} \AddNode\\
%  \intertext{\hfil isolati i termini con l'incognita \hfil}
6x-18 = 3x-3+\overbrace{5(\dfrac{5x+2}{5})}-2x-17\AddNode\\
6x-18 = 3x-3+5x+2-2x-17  \AddNode\\
  \intertext{\hfil isolati i termini con l'incognita \hfil}
 \overbrace{6x-3x-5x+2x}=\overbrace{18-3+2-17}  \AddNode\\
 0=0\AddNode
  \end{align*}
  \LinkNodes{Precedenze}
   \LinkNodes{Moltiplico}
  \LinkNodes{Separo}
  \LinkNodes{Sommo}
  \end{NodesList}
  Il primo membro è uguale al secondo l'uguaglianza è sempre vera. 







%\begin{table}[H]
%\centering
%\begin{tabular}{LCR}
%\toprule
%+a&=&\ldots\\
%\ldots&=&-a\\
%\bottomrule
%\end{tabular}
%\caption{Regola del trasporto}
%\label{tab:regtrasporto}
%\end{table}
%\begin{table}[H]
%\centering
%\begin{tabular}{LCR}
%\toprule
%\dfrac{\cdots\cdots}{a}&=&\dfrac{\cdots\cdots}{a}\\
%&\\
%a\cdot\dfrac{\cdots\cdots}{a}&=&a\cdot\dfrac{\cdots\cdots}{a}\\
%&\\
%\cdots\cdots&=&\cdots\cdots\\
%\bottomrule
%\end{tabular}
%\caption{Semplificazione denominatore}
%\label{tab:Semplificazionedenominatore}
%\end{table}
%\begin{table}[H]
%
%\centering
%\begin{tabular}{CCCCL}
%\toprule
%\multicolumn{5}{c}{ax=b}\\
%\hline
%%&\\
%\multicolumn{2}{c}{coefficienti}&&soluzione&tipo soluzione\\
%\midrule
%a\neq0&b\neq0&ax=b&x=\dfrac{b}{a}&determinata\\
%%&\\
%a\neq0&b=0&ax=0&x=0&determinata\\
%%&\\
%a=0&b=0&0x=0&&indeterminata\\
%%&\\
%a=0&b\neq0&0x=b&&impossibile\\
%\bottomrule	
%\end{tabular}
%\caption{Soluzioni equazioni primo grado intere}
%\label{tab:equazioniprimogrado}
%\end{table}
%\begin{table}%
%
%\centering
%\begin{tabular}{LR}
%\toprule
%Tipo&Nome\\
%\midrule
%ax^2+c=0&Pura\\
%\hline
%\multicolumn{2}{c}{Risoluzione}\\
%\multicolumn{2}{C}{ax^2=-c}\\
%\multicolumn{2}{C}{x^2=-\dfrac{c}{a}}\\
%\multirow{3}*{Se $-\dfrac{c}{a}>0$ esistono soluzioni reali} &x_1=-\sqrt{-\dfrac{c}{a}}\\
%&\\
%&x_2=+\sqrt{-\dfrac{c}{a}}\\
%&\\
%Se -\dfrac{c}{a}<0\text{ non esistono soluzioni reali}&\\
%&\\
%\bottomrule	
%%\end{tabular}
%%\caption{Equazione secondo grado pura}
%%\label{tab:equazione2GradoPura}
%%\end{table}
%%\begin{table}%
%%
%%\centering
%%\begin{tabular}{LR}
%\toprule
%Tipo&Nome\\
%\midrule
%ax^2+bx=0&Spuria\\
%\hline
%\multicolumn{2}{c}{Risoluzione}\\
%\multicolumn{2}{C}{ax^2+bx=0}\\
%\multicolumn{2}{C}{x(ax+b)=0}\\
%\multicolumn{2}{C}{x_1=0}\\
%\multicolumn{2}{C}{ax+b=0}\\
%\multicolumn{2}{C}{x_2=-\dfrac{b}{a}}\\
%\bottomrule	
%%\end{tabular}
%%\caption{Equazione secondo grado spuria}
%%\label{tab:equazione2GradoSpuria}
%%\end{table}
%%\begin{table}%
%%
%%\centering
%%\begin{tabular}{LR}
%\toprule
%Tipo&Nome\\
%\midrule
%ax^2=0&Monomia\\
%\hline
%\multicolumn{2}{c}{Risoluzione}\\
%\multicolumn{2}{C}{ax^2=0}\\
%\multicolumn{2}{C}{x_1=0}\\
%\multicolumn{2}{C}{x_2=0}\\
%\bottomrule	
%%\end{tabular}
%%\caption{Equazione secondo grado monomia}
%%\label{tab:equazione2GradoMonomia}
%%\end{table}
%%\begin{table}%
%%
%%\centering
%%\begin{tabular}{LR}
%\toprule
%Tipo&Nome\\%
%\midrule
%ax^2+bx+c=0&Completa\\%
%\hline
%\multicolumn{2}{c}{Risoluzione}\\%
%\multirow{3}*{$b^2-4ac>0$}&x_1=\dfrac{-b+\sqrt{b^2-4ac}}{2a}\\%
%&\\
%&x_2=\dfrac{-b-\sqrt{b^2-4ac}}{2a}\\%
%\hline
%\multirow{3}*{$b^2-4ac=0$}&x_1=-\dfrac{b}{2a}\\%
%&\\
%&x_2=-\dfrac{b}{2a}\\%
%\hline
%\multirow{3}*{$b^2-4ac<0$}&\\
%&\text{nessuna soluzione reale}\\%
%&\\
%\bottomrule	
%\end{tabular}
%\caption{Equazioni secondo grado}
%\label{tab:equazione2Gradoelenco}
%\end{table}

 \chapter{Equazioni frazionarie di primo grado}
\label{cha:Equazionefrazionariaprimogrado}
\section{Definizioni}
\label{sec:definizioni}
\begin{definizionet}{}{}
Un'equazione è frazionaria\index{Equazione!frazionaria} se l'incognita compare al denominatore.
\end{definizionet}
\begin{esempiot}{}{}
Classificare le seguenti equazioni
\end{esempiot}
L'equazione seguente non è frazionaria
\[3(x+1)+(x-2)(x-\dfrac{1}{2}=0) \]
mentre
\[\dfrac{3x+2}{4x+2}+x=0 \]
è un'equazione frazionaria l'incognita è al denominatore.

Una frazione è una divisione e in una divisione non è possibile dividere per zero. Può accadere  che vi siano valori dell'incognita che rendono il denominatore uguale a zero e di conseguente impossibile la frazione. Il dominio di una funzione o campo di esistenza è l'insieme dei valori per cui la frazione esiste.
\begin{procedurat}{}{}
\begin{enumerate}
\item Per ogni frazione che contengono le incognite discuto i denominatori.
	\begin{itemize}
	\item Pongo i denominatori uguali a zero e risolvo l'equazione che ottengo.
	\item Escludo i valori trovati negandoli $\neq$
	\end{itemize}
	\item  Scompongo il fattori primi i denominatori (attenzione alla differenza fra fattori ed addendi) es: $2x$ e  $2x+1$ sono due fattori fra loro diversi.
	\item Calcolo il mcm (Fattori comuni e non comuni, presi una sola volta con il massimo esponente)
	\item Traccio la linea di frazione 
	\item Per ogni frazione divido il minimo comune multiplo per il denominatore e il risultato della divisione lo moltiplico per il numeratore ricordando che sono obbligatorie le parentesi quando 
	\begin{itemize}
	\item Moltiplico fra loro polinomi
	\item Davanti alla linea di frazione vi è un segno meno
	\end{itemize}
	\item Ottengo un unica frazione che semplifico togliendo il denominatore
	\item Eseguo i calcoli e separo le incognite che scrivo sinistra, dai numeri che scrivo a destra, ricordando che  se un termine viene spostato rispetto all'uguale cambia di segno. Attenzione Se un numero moltiplica una lettere es $2x$ è un'incognita e andrà a sinistra ,diverso da $2$ che andra a destra.
	\item Sommo fra di loro le incognite e fra di loro i numeri.
	\item Ottengo 
	\begin{itemize}
	\item Un'equazione di primo grado che risolvo dividendo  a sinistra e a destra per il numero davanti all'incognita. Attenzione ogni numero ha un segno che non può essere trascurato.
	\begin{itemize}
	\item Controllo se i risultati ottenuti  sono accettabili confrontandoli con i valori che ho escluso eventualmente scartandoli nel caso fossero uguali.
	\end{itemize}
	\item Un'uguaglianza impossibile. Esempio $0=2$
	\item Un'identità. Esempio $2=2$
	\end{itemize}
\end{enumerate}
\end{procedurat}

 \include{quarto/disequazioni_primogrado}
\chapter{Sistemi lineari in due incognite}
\label{sec:sistemiLineariInDueIncognite}
\section{Equazioni in più variabili}
\label{sec:EquazioniInPiuVariabili}
\begin{definizionet}{Equazione in più variabili}{}\index{Equazione!più variabili!definizione}
	Un'equazione in più variabili è un'equazione in più incognite.
\end{definizionet}
\begin{definizionet}{Definizione di soluzione}{}\index{Equazione!più variabili!soluzione}
Per un'equazione in più variabili, una soluzione, è un insieme ordinato di valori che verificano l'equazione.
\end{definizionet}
\begin{esempiot}{}{}
	Risolviamo l'equazione\[ 3x+y=7\]
\end{esempiot}	
Questa equazione, in due incognite, ha per soluzione
$x=2$ e $y=1$ infatti \[3\cdot2+1=7\]
Ma anche $x=3$ e $y=-2$ infatti \[3\cdot3-2=7\] è soluzione.
In genere un'equazione in più incognite ha più di una soluzione.
\begin{figure}
	\centering
\includestandalone[width=.9\textwidth]{secondo/sistemi/mappe_concettuali_sistema_1}
	\caption{Classificazione di un sistema}
	\label{fig:ClassificazioneDiUnSistema}
\end{figure}
\section{Sistemi}
\label{sec:Sistemi}
Più  equazioni  possono avere le stesse soluzioni. Un sistema è un'insieme di due o più equazioni. Risolvere un sistema è verificare che più equazioni hanno le stesse soluzioni.
\begin{definizionet}{Definizione di sistema}{}\index{Sistema!definizione}
Un sistema è un'insieme di due o più  equazioni.
\end{definizionet}
\begin{definizionet}{Definizione di grado di sistema}{}\index{Sistema!grado}\label{def:sistemaGrado}
Il grado di un sistema, di più equazioni in tante incognite,  è il prodotto dei gradi delle equazioni del sistema ridotte in forma normale.
\end{definizionet}
\begin{definizionet}{Definizione di soluzione}{}\index{Sistema!definizione!soluzione}
Una soluzione per un sistema è un insieme ordinato di valori che sono soluzione per ogni equazione del sistema.
\end{definizionet}
\section[Classificazione rispetto alle soluzioni]{Classificazione dei sistemi rispetto alle soluzioni}
\begin{definizionet}{Definizione di equivalenza}{}\index{Sistema!definizione di equivalenza}
Due sistemi sono equivalenti quando hanno lo stesso insieme soluzione.
\end{definizionet}
\begin{definizionet}{Forma normale di un sistema lineare}{}\index{Sistema!lineare!forma normale}{}
Un sistema di due equazioni di 1 grado in due incognite x, y, a coefficienti numerici, si dice ridotto in forma normale se è del tipo
\[\left\{\begin{array}{l} {ax+by=c} \\ {a'x+b'y=c'}\end{array}\right. \]
dove a, b, a', b' si chiamano coefficienti delle incognite c, c' si chiamano termini noti
\end{definizionet}
\begin{teoremat}{Teorema fondamentale}{}\index{Sistema!lineare!teorema fondamentale}{}
Se i coefficienti del sistema lineare sono diversi da zero e non sono tra loro proporzionali $\dfrac{a}{a'} \ne \dfrac{b}{b'}$ il sistema ammette una e una sola soluzione data da
\[
\begin{cases}
	x=\dfrac{b'c-bc'}{ab'-a'b}\\
	y=\dfrac{ac'-a'c}{ab'-a'b}
\end{cases}
\]
Se, invece, sono proporzionali solo i coefficienti delle incognite cioè; $\dfrac{a'}{a} =\dfrac{b'}{b} \ne \dfrac{c'}{c} $, allora il sistema è impossibile
Se, invece, sono proporzionali i coefficienti e i termini noti delle equazioni cioè; $\dfrac{a'}{a} =\dfrac{b'}{b} =\dfrac{c'}{c} $, allora il sistema è indeterminato.
\end{teoremat}
Consideriamo il sistema canonico, \[ \left\{\begin{array}{l} {ax+by=c} \\ {a'x+b'y=c'} \end{array}\right. \]
moltiplicando la prima equazione per b' e la seconda per --b, avremo  \[\left\{\begin{array}{l} {ab'x+bb'y=b'c} \\ {-ba'x-bb'y=-bc'} \end{array}\right. \]
sommando lungo le colonne otteniamo \[\dfrac{\left\{\begin{array}{l} {ab'x+bb'y=b'c} \\ {-ba'x-bb'y=-bc'} \end{array}\right. }{\left(ab'-ba'\right)x=b'c-bc'} \]  da cui  \[x=\dfrac{b'c-bc'}{ab'-a'b} \]
moltiplicando la prima equazione per a' e la seconda per --a avremo  \[\left\{\begin{array}{l} {aa'x+ba'y=a'c} \\ {-aa'x-ab'y=-ac'} \end{array}\right. \] sommando lungo le colonne otteniamo  \[\dfrac{\left\{\begin{array}{l} {aa'x+ba'y=a'c} \\ {-aa'x-ab'y=-ac'} \end{array}\right. }{\left(a'b-ab'\right)y=a'c-ac'} \] da cui \[y=\dfrac{ac'-a'c}{ab'-a'b} \]  da cui se  \[\dfrac{a'}{a} \ne \dfrac{b'}{b} \] avremo  \[ab'-a'b\ne 0\]  quindi il sistema ha una e una sola soluzione.
Se \[\dfrac{a'}{a} =\dfrac{b'}{b} \]  avremo due casi  \[\dfrac{a'}{a} =\dfrac{b'}{b} \ne \dfrac{c'}{c} \]  e  \[\dfrac{a'}{a} =\dfrac{b'}{b} =\dfrac{c'}{c} \]
Nel primo caso il sistema è impossibile perché  $ab'-a'b=0$, $a'c-ac'\ne 0$, $b'c-bc'\ne 0$  e quindi poiché è possibile riscrivere il sistema canonico nella forma  \[\left\{\begin{array}{l} {\left(ab'-ab'\right)x=b'c-bc'} \\ {\left(ab'-ab'\right)y=ac'-a'c} \end{array}\right. \]  da cui  \[\left\{\begin{array}{l} {0=b'c-bc'} \\ {0=ac'-a'c} \end{array}\right. \] impossibile.
Nel secondo caso il sistema è indeterminato perché  $ab'-a'b=0$, $a'c-ac'=0$, $b'c-bc'=0$  e quindi giacché è possibile riscrivere il sistema canonico nella forma  \[\left\{\begin{array}{l} {\left(ab'-ab'\right)x=b'c-bc'} \\ {\left(ab'-ab'\right)y=ac'-a'c} \end{array}\right. \]  da cui \[\left\{\begin{array}{l} {0=0} \\ {0=0} \end{array}\right. \]  indeterminato
\section{Metodi di risoluzione}
\label{sec:MetodiDiRisoluzione}
\subsection{Sostituzione}
\label{sec:Sostituzione}
Un sistema lineare in forma canonica si risolve con il metodo di sostituzione\index{Sistema!metodo!sostituzione} isolando una variabile in una equazione e sostituendola nelle altre. 
\begin{esempiot}{}{}
Supponiamo di avere un sistema lineare in forma normale
\[
\begin{cases}
	x+2y=1\\
	3x-y=2
\end{cases}
\]
\end{esempiot}
risolvo la prima rispetto alla x e ottengo 
\[
\begin{cases}
	x=1-2y\\
	3x-y=2
\end{cases}
\]
sostituisco nella seconda ed ottengo
\[
\begin{cases}
	x=1-2y\\
	3(1-2y)-y=2
\end{cases}
\begin{cases}
	x=1-2y\\
	3-6y-y=2
\end{cases}
\begin{cases}
	x=1-2y\\
	-7y=2-3
\end{cases}
\begin{cases}
	x=1-2y\\
	-7y=-1
\end{cases}
\begin{cases}
	x=1-2y\\
	y=\dfrac{1}{7}
\end{cases}
\begin{cases}
	x=1-2\dfrac{1}{7}\\
	y=\dfrac{1}{7}
\end{cases}
\]
\[
\begin{cases}
	x=\dfrac{7-2}{7} \\
	y=\dfrac{1}{7}
\end{cases}
\begin{cases}
	x=\dfrac{5}{7}\\
	y=\dfrac{1}{7}
\end{cases}
\]
\subsection{Confronto}
\label{sec:Confronto}
Un sistema lineare in forma canonica si risolve con il metodo del confronto\index{Sistema!metodo!confronto} risolvendo due equazioni rispetto ad un stessa variabile e confrontando i valori attenuti.
\begin{esempiot}{}{}
supponiamo di avere un sistema lineare in forma normale
\[
\begin{cases}
	x+2y=1\\
	3x-y=2
\end{cases}
\]
\end{esempiot}
Risolvo rispetto ad x e ottengo
\[
\begin{cases}
	x=-2y+1\\
	x=\dfrac{2+y}{3}
\end{cases}
\begin{cases}
	-2y+1=\dfrac{2+y}{3}\\
		x=-2y+1
\end{cases}
\begin{cases}
	-6y+3=2+y\\
	x=-2y+1
\end{cases}
\begin{cases}
	-7y=-1\\
	x=-2y+1
\end{cases}
\begin{cases}
	y=\dfrac{1}{7}\\
	x=-2y+1
\end{cases}
\]
\[
\begin{cases}
	y=\dfrac{1}{7}\\
	x=-2\dfrac{1}{7}+1
\end{cases}
\begin{cases}
	y=\dfrac{1}{7}\\
	x=\dfrac{5}{7}
\end{cases}
\]

\subsection{Riduzione}
\label{sec:Riduzionesist}
Questo metodo è chiamato anche di somma sottrazione e consiste nel sommare o sottrarre le equazioni in modo che vengano determinate le incognite. Iniziamo con un po di vocabolario, chiamiamo riga un'equazione del sistema. Mentre la colonna è formate in verticale dalla stessa incognita nelle varie equazioni.
\begin{esempiot} {}{}
	Risolvere il sistema \[
	\begin{cases}
	2y+3x=2\\
	x+y=5
	\end{cases}\]
\end{esempiot}
Il sistema è formato da due equazioni, quindi da due righe. Il sistema non è ordinato per colonne quindi bisogna riscriverlo in questo modo \[
\begin{cases}
3x+2y=2\\
x+y=5
\end{cases}\]
Il metodo consiste nel moltiplicare le righe per dei valori  opportuni e  sommare o sottrarre lungo le colonne in modo che una variabile scompaia.
\[
\begin{cases}
3x+2y=2\\
\tikzmark{1}3x+3y=15\tikzmark{2}
\end{cases}\]
\begin{tikzpicture}[remember picture, overlay]
\node[below=0.1cm and 0cm of 1](3){};
\node[below=0.1cm and 0cm of 2](4){};
\node[left=2cm and 0cm of 1](5){$3$};
\node[below right=0.2cm and 0.1cm of 5](6) {$0-y=-13$};
\draw (3) edge  (4);
\end{tikzpicture}\\

Moltiplicando la seconda riga per tre e sottraendola alla prima otteniamo che $y=13$. 

\[
\begin{cases}
3x+2y=2\\
\tikzmark{1}2x+2y=10\tikzmark{2}
\end{cases}\]
\begin{tikzpicture}[remember picture, overlay]
\node[below=0.1cm and 0cm of 1](3){};
\node[below=0.1cm and 0cm of 2](4){};
\node[left=2cm and 0cm of 1](5){$2$};
\node[below right=0.2cm and 0.1cm of 5](6) {$x+0 =-8$};
\draw (3) edge  (4);
\end{tikzpicture}\\

Ripartendo dal sistema iniziale e moltiplicando la seconda riga per due otteniamo che $x=-8$

     
	\chapter{Formule inverse}
\label{cha:semplificazioni}
\section[Primo caso]{Primo caso $a\cdot c=b$}
\label{sec:primocasosemp}
Problema: dato $a\cdot c=b$ trovare $c$.
\begin{align*}
a\cdot c=b&&\\
\text{Divido per $a$ entrambi i lati dell'uguaglianza, e ottengo}\\
\dfrac{a\cdot c}{a}=\dfrac{b}{a}&&\\
\text{Semplifico a sinistra e ottengo c}\\
c=\dfrac{b}{a}&&
\end{align*}
\section[Secondo caso]{Secondo caso $a=\dfrac{b}{c}$}
\label{sec:secondocasosemp}
Problema: dato $a=\dfrac{b}{c}$ trovare $c$.
\begin{align*}
a=\dfrac{b}{c}&&\\
\text{Moltiplico per $c$ entrambi i lati dell'uguaglianza ed ottengo }\\
a\cdot c=\dfrac{b}{c}\cdot c&&\\
\text{Semplifico a destra e ottengo}\\
a\cdot c=b&&\\
\text{Quindi procedo come in\nobs\ref{sec:primocasosemp} }
\end{align*}
\section[Terzo caso]{Terzo caso $a=\dfrac{b}{c+d}$}
\label{sec:terzocasosemp}
\subsection{Trovare b}
\label{sec:terzocasosemptrovareb}
Problema: dato $a=\dfrac{b}{c+d}$ trovare $b$
\begin{align*}
a=\dfrac{b}{c+d}&&\\
\text{Moltiplico per $c+d$ entrambi i lati dell'uguaglianza ed ottengo }\\
a\cdot (c+d)=\dfrac{b}{c+d}\cdot(c+d)&&\\
\text{Semplifico a destra e ottengo}\\
a\cdot (c+d)=b&&
\end{align*}
\subsection{Trovare d}
\label{sec:terzocasosemptrovared}
Problema: dato $a=\dfrac{b}{c+d}$ trovare $d$
\begin{align*}
a=\dfrac{b}{c+d}&&\\
\text{Moltiplico per $c+d$ entrambi i lati dell'uguaglianza ed ottengo }\\
a\cdot (c+d)=\dfrac{b}{c+d}\cdot(c+d)&&\\
\text{Semplifico a destra e ottengo}\\
a\cdot (c+d)=b&&\\
ac+ad=b&&\\
ad=-ac+b&&\\
\text{Divido per a}\\
d=\dfrac{-ac+b}{a}
\end{align*}
\section[Quarto caso]{Quarto caso $\dfrac{1}{a}=\dfrac{b+ c}{b\cdot c}$}
\label{sec:quartocasosemp}
\subsection{Trovare b}
\label{sec:quartocasosemptrovareb}
Problema: dato $\dfrac{1}{a}=\dfrac{b+ c}{b\cdot c}$
\begin{align*}
\dfrac{1}{a}=\dfrac{b+ c}{b\cdot c}&&\\
\text{calcolo il m.c.m }\\
\dfrac{b\cdot c=a\cdot(b+c)}{a\cdot b\cdot c}&&\\
\text{tolgo il m.c.m }\\
b\cdot c=a\cdot(b+c)&&\\
b\cdot c=a\cdot b+a\cdot c &&\\
b\cdot c-a\cdot b=a\cdot c &&\\
b\cdot (c-a)=a\cdot c &&\\
b=\dfrac{a\cdot c}{c-a}
\end{align*}

	\chapter{Radicali}
\label{Radicaliradici}
\section{Glossario}
\begin{table}[H]
\centering
$\sqrt[n]{a^m}=b$
\begin{itemize}
\item n Indice del radicale
\item $a^m$ Radicando
\item m Esponente o potenza del radicando
\item b Radice
\end{itemize}
\caption{Glossario}
\label{tab:RadicaliGlossario}
\end{table}
\begin{table}[H]
\centering
$
\begin{array}{rccc}
\toprule
 &\text{Indice} & \text{Potenza} & \text{Radicando} \\ 
 \midrule
 \sqrt[3]{a}& 3 &1  & a \\ 
 \sqrt[4]{a^3b}& 4 &1  & a^3b \\ 
 \sqrt[4]{a^2}& 4 &2  & a^2 \\
\sqrt{a^5}& 2 &5 & a^5 \\ 
\sqrt[2]{a^5}& 2 &5 & a^5 \\ 
\sqrt[4]{\dfrac{\left( a+b\right)^2 }{c}}& 4 &1 &\dfrac{\left( a+b\right)^2 }{c} \\
\sqrt[3]{a^2+b}& 3 &1 & a^2+b\\
\bottomrule	
\end{array}
$ 
\label{tab:esempiglossario}
\caption{Esempi Glossario}
\end{table}

\begin{table}[H]
\centering
\begin{itemize}
	\item $\sqrt[1]{a}=a\; \forall a\in\R^{+}\forall\; n\in\Ni$
	\item $\sqrt[n]{a^n}=a\; \forall a\in\R^{+}\forall\; n\in\Ni$
	\item $\left(\sqrt[n]{a}\right)^n=a\; \forall a\in\R^{+}\forall\; n\in\Ni$
	\item $\sqrt[nk]{a^{mk}}=\sqrt[n]{a^m}\;  \forall a\in\R^{+}\forall\; n,m,k\in\Ni$\label{Rad:invariantiva}\index{Radicali!proprietà invariantiva}
\end{itemize}
\label{tab:propRadicli}
\caption{Proprietà dei radicali}
\end{table}
\section{Proprietà invariantiva}
\label{Sec:Propinvariantivaradicali}
\begin{definizionet}
Si ottiene una radice equivalente moltiplicando o dividendo la potenza e l'indice per lo stesso valore diverso da zero.
\end{definizionet}
\subsection{Riduzione allo stesso indice}
\label{sec:RiduzioneAlloStessoIndice}

La Proprietà invariantiva permette di ridurre due radici allo stesso indice. Procediamo in questo modo
\begin{enumerate}
	\item Calcolo in m.c.m fra gli indici di tutte le radici
	\item Scrivo delle nuove radici di indice uguale al m.c.m. e per ogni radice
	\begin{itemize}
	\item Divido il m.c.m per l'indice  della radice e moltiplico il numero ottenuto per l'esponente m del radicando della  radice.
	\end{itemize}
\end{enumerate}

\begin{table}[H]
\centering
$
\begin{array}{cccl}
\toprule
\text{Passo} &  &  & \text{Note} \\  
\midrule
0 & \sqrt[3]{a} &\sqrt[5]{b}  &\text{Inizio} \\ 
1 & 15\div 3=5 &15\div 5=3  & mcm(3,5)=15 \\  
2 &\sqrt[15]{a^{1\cdot5}}=\sqrt[15]{a^{5}}  &\sqrt[15]{b^{1\cdot3}}=\sqrt[15]{b^{1\cdot3}}  &\text{Fine} \\
\bottomrule	
\end{array} 
$
\label{tab:Es1Ridstessoindice}
\caption{Esempio riduzione stesso indice}
\end{table}
\begin{table}[H]
\centering
$
\begin{array}{cccl}
\toprule
\text{Passo} &  &  &\text{Note} \\  
\midrule
0 & \sqrt{ab^2} &\sqrt[5]{a+b^2}  &\text{Inizio} \\ 
1 & 10\div 2=5 &10\div 5=2  & mcm(2,5)=10 \\  
2 &\sqrt[10]{a^{1\cdot 5}b^{2\cdot 5}}=\sqrt[10]{a^{2}b^{10}}& \sqrt[10]{\left( a+b^2\right)^{1\cdot 2} }=\sqrt[10]{\left( a+b^2\right)^{2} }   &\text{Fine} \\
\bottomrule	
\end{array} 
$
\label{tab:Es1Ridstessoindice2}
\caption{Esempio riduzione stesso indice}
\end{table}
\subsection{Ordinamento fra radici}
\label{sec:OrdinamentoFraRadici}
Il procedimento di riduzione allo stesso indice permette di confrontare due radici di indice diverso. Basta, dopo aver ridotto le radici allo stesso indice, confrontare i radicandi.
\begin{table}[H]
\centering
$
\begin{array}{ccccl}
\toprule
\text{Passo} &  &  &  &\text{Note} \\ 
\midrule
0 & \sqrt{5} &  &\sqrt[3]{6} & \text{Inizio} \\ 
1 & \sqrt[6]{5^3} &  &\sqrt[6]{6^2}  &\text{Riduzione stesso indice} \\ 
2 & 5^3=125 & > & 6^2=36 & \text{Confronto fra radicandi} \\ 
3 & \sqrt{5} & > & \sqrt[3]{6} & \text{Fine} \\
\bottomrule	
\end{array} 
$
\label{tab:confrontoradicali}
\caption{Esempio confronto radicali}
\end{table}

\subsection{Semplificare radici}
\label{sec:RadiciIriducibili}
La proprietà invariantiva permette di semplificare l'indice e l'esponente  di una radice.
Una radice è irriducibile se non è possibile semplificare l'indice con l'esponente\footnote{Cioè quando l'indice e l'esponente sono primi fra loro}
\begin{table}[H]
\centering
$
\begin{array}{ccl}
\toprule
\text{Passo} &  & \text{Note} \\ 
\midrule
0 &\sqrt[10]{a^2b^4} & \text{Inizio} \\ 
1 &  &  10,2,4 \text{ si dividono per }2\\ 
2 &\sqrt[5]{ab^2}  &\text{Fine}\\
\bottomrule
\end{array} 
$
\label{Tab:radiceriducibile}
\caption{Esempio radice riducibile}
\end{table}
\begin{table}[H]
\centering
$
\begin{array}{ccl}
\toprule
\text{Passo} &  & \text{Note} \\ 
\midrule
0 &\sqrt[10]{a^2+b^4} & \text{Inizio} \\ 
1 &  &  10,2,4 \text{ si dividono per }2 \text{ ma è una somma}\\ 
2 &\sqrt[10]{a^2+b^4}  &\text{Fine}\\
\bottomrule
\end{array} 
$
\label{Tab:radiceriducibilece}
\caption{Esempio radice irriducibile}
\end{table}
\begin{table}[H]
\centering
$
\begin{array}{ccl}
\toprule
\text{Passo} &  & \text{Note} \\ 
\midrule
0 &\sqrt[20]{\dfrac{a^5\left( a^5+b\right)^{10} }{x^{15}}} & \text{Inizio} \\ 
1 &  &  20,10,5,15 \text{ si dividono per }5 \\ 
2 &\sqrt[4]{\dfrac{a\left( a^5+b\right)^{2} }{x^{3}}} & \text{Fine} \\ 
\bottomrule
\end{array} 
$
\label{Tab:radiceriducibilece2}
\caption{Esempio radice riducibile}
\end{table}
\begin{table}[H]
\centering
$
\begin{array}{ccl}
\toprule
\text{Passo} &  & \text{Note} \\ 
\midrule
0 &\sqrt[10]{a^{20}}= & \text{Inizio} \\ 
1 &  &  20,10 \text{ si dividono per }10 \\ 
2 &=\sqrt[1]{a^2}=a^2 & \text{Fine} \\ 
\bottomrule
\end{array} 
$
\label{Tab:radiceriducibilece3}
\caption{Esempio radice riducibile}
\end{table}
\begin{table}[H]
\centering
$
\begin{array}{ccl}
\toprule
\text{Passo} &  & \text{Note} \\ 
\midrule
0 &\sqrt[8]{(-2)^{20}}= & \text{Inizio} \\ 
1 &  &  8,10 \text{ si dividono per }2\\
& \sqrt[4]{(-2)^{5}}&\text{ ma il radicando è negativo, quindi non esiste radice reale} \\ 
 & & \text{Fine} \\ 
\bottomrule
\end{array} 
$
\label{Tab:radiceriducibilece4}
\caption{Esempio radice non riducibile}
\end{table}
\begin{table}[H]
\centering
$
\begin{array}{ccl}
\toprule
\text{Passo} &  & \text{Note} \\ 
\midrule
0 &\sqrt[10]{(-2)^{6}}= & \text{Inizio} \\ 
1 &  &  10,6 \text{ si dividono per }2\\
& \sqrt[5]{(-2)^{3}}&\text{il radicando è negativo ma l'indice è dispari quindi esiste la radice reale} \\ 
2 &=\sqrt[5]{(-2)^{3}} & \text{Fine} \\ 
\bottomrule
\end{array} 
$
\label{Tab:radiceriducibilece5}
\caption{Esempio radice riducibile}
\end{table}
\section{Operazioni con le Radici}
\label{sec:operazioniradici}
\subsection{Prodotto}
\label{sec:prodottoradici}
Il prodotto di due o più radici, con lo stesso indice, è una radice che ha per indice lo stesso indice e per radicando il prodotto dei radicandi.

\[\sqrt[n]{a}\cdot\sqrt[n]{b}=\sqrt[n]{a\cdot b}\]
\begin{table}[H]
\centering
$
\begin{array}{ccl}
\toprule
\text{Passo} &  & \text{Note} \\ 
\midrule
0 &\sqrt[3]{a^3b}\cdot\sqrt[3]{a^4b^2x}=  & \text{Inizio} \\ 
1 &  &  \text{i due radicali hanno lo stesso indice} \\ 
2 &=\sqrt[3]{a^3b}\cdot\sqrt[3]{a^4b^2x}=\sqrt[3]{a^7b^3x} & \text{Fine} \\ 
\bottomrule
\end{array} 
$
\label{Tab:radicprodotto1}
\caption{Esempio prodotto di radici con lo stesso indice}
\end{table}
Se le radici hanno indice diverso bisogna prima ridurle allo stesso indice.
\begin{table}[H]
\centering
$
\begin{array}{ccl}
\toprule
\text{Passo} &  & \text{Note} \\ 
\midrule
0 &\sqrt[3]{a^3b}\cdot\sqrt[5]{a^4b^2x}=  & \text{Inizio} \\ 
1 &  &   \text{I due radicali hanno indice diverso} \\ 
2 &=\sqrt[15]{a^{15}b^5}\cdot\sqrt[15]{a^4b^2x}= & \text{Riduco allo stesso indice} \\ 
3 &=\sqrt[15]{a^{15}b^5}\cdot\sqrt[15]{a^4b^2x}=\sqrt[15]{a^{17}b^{11}x^3} & \text{Fine} \\ 
\bottomrule
\end{array} 
$
\label{Tab:radiceprodotto2}
\caption{Esempio prodotto di radici con indice diverso}
\end{table}
\begin{table}[H]
\centering
$
\begin{array}{ccl}
\toprule
\text{Passo} &  & \text{Note} \\ 
\midrule
0 &\sqrt[3]{\dfrac{ya}{x}}\cdot\sqrt{\dfrac{x^2}{y}}\cdot\sqrt[6]{\dfrac{y}{x^4}}  & \text{Inizio} \\ 
1 &  &   \text{I due radicali hanno indice diverso} \\ 
2 &=\sqrt[15]{a^{15}b^5}\cdot\sqrt[15]{a^4b^2x}= & \text{Riduco allo stesso indice} \\ 
3 &=\sqrt[15]{a^{15}b^5}\cdot\sqrt[15]{a^4b^2x}=\sqrt[15]{a^{17}b^{11}x^3} & \text{Fine} \\ 
\bottomrule
\end{array} 
$
\label{Tab:radiceprodotto3}
\caption{Esempio prodotto di radici con indice diverso}
\end{table}
\subsubsection{Trasporto di un termine fuori del segno di radice}
\label{sec:Trasportofuoriradici}
Se la potenza m del radicando è maggiore o uguale all'indice n del radicando allora:
\[\sqrt[n]{a^m}=a^q\sqrt[n]{a^r}\] 
\begin{itemize}
\item n Indice radice
\item m Esponente o potenza del radicando
\item q Quoziente della divisione di m per n
\item r Resto della divisione di m per n\footnote{Per ottenere r basta moltiplicare la parte decimale della divisione di per n. Es $3\div 2=1,5$ $r=0,5\cdot 2=1$ }
\end{itemize}
\begin{table}[H]
\centering
$
\begin{array}{ccc}
\toprule
\text{Passo} &  & \text{Note} \\
\midrule
0& \sqrt[3]{32}=\sqrt[3]{2^5} & \text{Inizio} \\ 
1 & \text{Divido } 5 \text{  per } 3 &q=1\text{  }r=2  \\ 
2&  \sqrt[3]{32}=\sqrt[3]{2^5}=2^1\sqrt[3]{2^2}=2\sqrt[3]{4}& \text{Fine} \\ 
\bottomrule 
\end{array}
$ 
\label{tab:Trasportofuoriradici1}
\caption{Esempio trasporto di un termine fuori del segno di radice}
\end{table}
\begin{table}[H]
\centering
$
\begin{array}{ccc}
\toprule
\text{Passo} &  & \text{Note} \\
\midrule
0& \sqrt[8]{a^{15}} & \text{Inizio} \\ 
1 & \text{Divido } 15 \text{  per } 8 &q=1\text{  }r=7  \\ 
2&  \sqrt[8]{a^{15}}=a^1\sqrt[8]{a^7}=a\sqrt[8]{a^7}& \text{Fine} \\ 
\bottomrule 
\end{array}
$ 
\label{tab:Trasportofuoriradici2}
\caption{Esempio trasporto di un termine fuori del segno di radice}
\end{table}
\begin{table}[H]
\centering
$
\begin{array}{ccc}
\toprule
\text{Passo} &  & \text{Note} \\
\midrule
0& \sqrt{72}=\sqrt{3^2\cdot 2^3} & \text{Inizio} \\ 
1 & \text{Divido } 2 \text{  per } 2 q=1\text{  }r=0 \\
&\text{ Divido } 3 \text{  per } 2\text{  } q=1\text{  }r=1    \\ 
2&  \sqrt{72}=\sqrt{3^2\cdot 2^3}=3^1\cdot 2^1\sqrt{3^0 2}=6\sqrt{2}& \text{Fine} \\ 
\bottomrule 
\end{array}
$ 
\label{tab:Trasportofuoriradici3}
\caption{Esempio trasporto di un termine fuori del segno di radice}
\end{table}
\begin{table}
\centering
$
\begin{array}{ccc}
\toprule
\text{Passo} &  & \text{Note} \\
\midrule
0& \sqrt[3]{\dfrac{a^6b^2}{c^4\left( a+b\right)^3 }} & \text{Inizio} \\ %
1 & considero gli esponenti maggiori o uguali all'indice della radice cioè 6,4,3 \\
&\text{ Divido } 6 \text{  per } 3\text{  } q=2\text{  }r=0    \\ 
&\text{ Divido } 4 \text{  per } 3\text{  } q=1\text{  }r=1    \\
&\text{ Divido } 3 \text{  per } 3\text{  } q=1\text{  }r=0    \\
2& \sqrt[3]{\dfrac{a^6b^2}{c^4\left( a+b\right)^3 }}=\dfrac{a^2}{c\left(a+b\right)}\sqrt[3]{\dfrac{a^0b^2}{c^1\left( a+b\right)^0 }} =\dfrac{a^2}{c\left(a+b\right)  }\sqrt[3]{\dfrac{b^2}{c}} & \text{Fine} \\ 
\bottomrule 
\end{array}
$ 
\label{tab:Trasportofuoriradici4}
\caption{Esempio trasporto di un termine fuori del segno di radice}
\end{table}
\subsubsection{Trasporto di un termine dentro il segno di radice}
\label{sec:Trasportodentroradici}
\[b^p\sqrt[n]{a^m}=\sqrt[n]{b^{p\cdot n}a^m}\]
\begin{table}[H]
\centering
$
\begin{array}{ccc}
\toprule
\text{Passo} &  & \text{Note} \\
\midrule
0& 3\sqrt[2]{2} & \text{Inizio} \\ 
1& 3\sqrt[2]{2}=\sqrt{3^2\cdot 2}=\sqrt[2]{18} & \text{Fine} \\ 
\bottomrule 
\end{array}
$ 
\label{tab:Trasportodentroradici1}
\caption{Esempio trasporto di un termine dentro il segno di radice}
\end{table}
\begin{table}[H]
\centering
$
\begin{array}{ccc}
\toprule
\text{Passo} &  & \text{Note} \\
\midrule
0& \dfrac{c}{\left( a+b\right)^2 }\sqrt[3]{c} & \text{Inizio} \\ 
1& \dfrac{c}{\left( a+b\right)^2 }\sqrt[3]{c}=\sqrt[3]{\dfrac{c^{1\cdot 3}c}{\left(a+b \right) }^{2\cdot 3}}=\sqrt[3]{\dfrac{c^{4}}{\left(a+b \right) }^{6}} & \text{Fine} \\ 
\bottomrule 
\end{array}
$ 
\label{tab:Trasportodentroradici2}
\caption{Esempio trasporto di un termine dentro il segno di radice}
\end{table}
\section{Potenze}
\label{sec:PotenzeRadici}
\[\left( \sqrt[n]{a}\right)^m=\sqrt[n]{a^m}\]
\begin{table}[H]
\centering
$
\begin{array}{cc}
\toprule
\left( \sqrt[4]{a}\right)^3=\sqrt[4]{a^3} \\ 
\left( \sqrt[3]{a+b}\right)^2=\sqrt[2]{\left( a+b\right)^2 } \\ 
\left( \sqrt[5]{a^2b}\right)^3=\sqrt[4]{a^6b^3} \\ 
\bottomrule 
\end{array}
$ 
\label{tab:potenzeradici1}
\caption{Esempi potenze radicali}
\end{table}
\section{Quoziente} 
\label{sec:quozienteradicali}
\[\dfrac{\sqrt[n]{a}}{\sqrt[n]{b}}=\sqrt[n]{\dfrac{a}{b}}\]
\section{Radice di radice}
\label{sec:radicediradice}
\[\sqrt[n]{\sqrt[p]{a}}=\sqrt[n\cdot p]{a}\]
\section{Somma}
\label{sec:SommaReali}
\begin{itemize}
\item Due radici sono simile se hanno lo stesso indice e  lo stesso radicando.
\item Se consideriamo una radice come un monomio avremo che il numero che compare davanti al simbolo di radice è la "`parte numerica' mentre la radice è la "`parte letterale'. Possiamo dare la seguente definizione: La somma di due radici simili è una radice simile alle precedenti che ha per parte numerica la somma algebrica delle parti numeriche.
\end{itemize}
\section{Razionalizzazione del denominatore}
\label{sec:razzionalizzazionedenominatoreradici}
\subsection{Primo caso}
\bassapriorita{Inserire razionalizzazioni}
\label{sec:razionzinalizzadioneden1caso}
\subsection{Secondo caso}
\label{sec:razionzinalizzadioneden2caso}



	\chapter{Equazioni secondo grado}
\label{cha:equazioni2grado}
\section{Equazioni di secondo grado pure}
\begin{definizionet}{Equazione di secondo grado pura}{}
	Un'equazione\index{Equazione} di secondo grado è un'equazione del tipo\[ ax^2+c=0\] $a,c\in\R\;a\neq 0$
\end{definizionet}
\begin{esempiot} {}{}
	Sono equazioni di secondo grado pure le seguenti:
      \[2x^2+3=0 \]
      \[3x^2-4=0 \]
\end{esempiot}
\section{Equazioni di secondo grado spurie}
\begin{definizionet}{Equazione di secondo grado spuria}{}
	Un'equazione\index{Equazione} di secondo grado è un'equazione del tipo\[ ax^2+bx=0\] $a,b\in\R\;a\neq 0$
\end{definizionet}
\section{Equazioni di secondo grado monomie}
\begin{definizionet}{Equazione di secondo grado monomia}{}
	Un'equazione\index{Equazione} di secondo grado è un'equazione del tipo\[ ax^2=0\] $a\in\R\;a\neq 0$
\end{definizionet}
\section{Equazioni di secondo grado complete}
\begin{definizionet}{Equazione di secondo grado completa}{}
	Un'equazione\index{Equazione} di secondo grado è un'equazione del tipo\[ ax^2+bx+c=0\] $a,b,c\in\R\;a\neq 0$
\end{definizionet}

\begin{figure}
\centering
%\includegraphics[scale=0.80]{secondo/equazioni2gradopdf-crop.pdf}
\includestandalone[width=.9\textwidth]{secondo/diagrammi/mappe_concettuali_equa_sgrado_1}
\caption{Equazioni secondo grado}
\label{fig:equazioni2gradocmap}
\end{figure}
\begin{table}%
\centering
\begin{tabular}{LR}
\toprule
Tipo&Nome\\
\midrule
ax^2+c=0&Pura\\
\hline
\multicolumn{2}{c}{Risoluzione}\\
\multicolumn{2}{C}{ax^2=-c}\\
\multicolumn{2}{C}{x^2=-\dfrac{c}{a}}\\
\multirow{3}*{Se $-\dfrac{c}{a}>0$ esistono soluzioni reali} &x_1=-\sqrt{-\dfrac{c}{a}}\\
&\\
&x_2=+\sqrt{-\dfrac{c}{a}}\\
&\\
Se -\dfrac{c}{a}<0\text{ non esistono soluzioni reali}&\\
&\\
\bottomrule	
%\end{tabular}
%\caption{Equazione secondo grado pura}
%\label{tab:equazione2GradoPura}
%\end{table}
%\begin{table}%
%
%\centering
%\begin{tabular}{LR}
\toprule
Tipo&Nome\\
\midrule
ax^2+bx=0&Spuria\\
\hline
\multicolumn{2}{c}{Risoluzione}\\
\multicolumn{2}{C}{ax^2+bx=0}\\
\multicolumn{2}{C}{x(ax+b)=0}\\
\multicolumn{2}{C}{x_1=0}\\
\multicolumn{2}{C}{ax+b=0}\\
\multicolumn{2}{C}{x_2=-\dfrac{b}{a}}\\
\bottomrule	
%\end{tabular}
%\caption{Equazione secondo grado spuria}
%\label{tab:equazione2GradoSpuria}
%\end{table}
%\begin{table}%
%
%\centering
%\begin{tabular}{LR}
\toprule
Tipo&Nome\\
\midrule
ax^2=0&Monomia\\
\hline
\multicolumn{2}{c}{Risoluzione}\\
\multicolumn{2}{C}{ax^2=0}\\
\multicolumn{2}{C}{x_1=0}\\
\multicolumn{2}{C}{x_2=0}\\
\bottomrule	
%\end{tabular}
%\caption{Equazione secondo grado monomia}
%\label{tab:equazione2GradoMonomia}
%\end{table}
%\begin{table}%
%
%\centering
%\begin{tabular}{LR}
\toprule
Tipo&Nome\\%
\midrule
ax^2+bx+c=0&Completa\\%
\hline
\multicolumn{2}{c}{Risoluzione}\\%
\multirow{3}*{$b^2-4ac>0$}&x_1=\dfrac{-b+\sqrt{b^2-4ac}}{2a}\\%
&\\
&x_2=\dfrac{-b-\sqrt{b^2-4ac}}{2a}\\%
\hline
\multirow{3}*{$b^2-4ac=0$}&x_1=-\dfrac{b}{2a}\\%
&\\
&x_2=-\dfrac{b}{2a}\\%
\hline
\multirow{3}*{$b^2-4ac<0$}&\\
&\text{nessuna soluzione reale}\\%
&\\
\bottomrule	
\end{tabular}
\caption{Equazioni secondo grado}
\label{tab:equazione2Gradoelenco}
\end{table}

	\chapter{Esempi}
	\section{Circuiti e reti}
	\label{sec:CircuitieReti}
	\begin{table}[H]
		\caption{In un circuito con due resistenze $R_1$ e $R_2$ in parallelo, trovare la formula che da $R_2$ note $R_1$ e $R_{eq}$}
		\label{tab:Trovarediffangoli}
		\begin{enumerate}
			\item Prerequisiti 
			\begin{itemize}
				\item mcm
				\item Equazioni di primo grado
				\item Resistenze in parallelo
				\item $\dfrac{1}{R_{eq}}=\dfrac{1}{R_1}+\dfrac{1}{R_2}+\cdots+\dfrac{1}{R_n}$			
			\end{itemize}
			\item Scopo: Determinare una resistenza note l'altra e la resistenza equivalente in un circuito in parallelo
			\item Testo: Determinare $R_1$ noti $R_2$ e $R_{eq}$
			\item Svolgimento: Si usa la formula che da la resistenza equivalente in parallelo.
			\begin{enumerate}
				\item $\dfrac{1}{R_{eq}}=\dfrac{1}{R_1}+\dfrac{1}{R_2}+\cdots+\dfrac{1}{R_n}$
				\item $\dfrac{1}{R_{eq}}=\dfrac{1}{R_1}+\dfrac{1}{R_2}$
				\item trovo mcm fra $R_1$, $R_2$ e $R_{eq}$
				\item $\dfrac{R_1\cdot R_2=R_{eq}\cdot (R_1+R_2)}{R_1\cdot R_2\cdot R_{eq}}$
				\item $R_1\cdot R_2=R_{eq}\cdot (R_1+R_2)$
				\item $R_1\cdot R_2=R_{eq}\cdot R_1+R_{eq}\cdot R_2$
				\item $R_1\cdot R_2-R_{eq}\cdot R_1=R_{eq}\cdot R_2$
				\item $R_1\cdot (R_2-R_{eq})=R_{eq}\cdot R_2$
				\item $R_1=\dfrac{R_{eq}\cdot R_2}{(R_2-R_{eq})}$
			\end{enumerate}
		\end{enumerate}
	\end{table}
 % % % % % % % % % % % %FINE SECONDO


	\backmatter
	\addcontentsline{toc}{chapter}{\indexname}
	\printindex
	\appendix
%	\input{terzo/Tabelle_goniometriche}
	\chapter{Mezzi usati}
	\CDMezziUsati
	
	
	
\end{document}
\DeclareCaptionFormat{grafico}{\textbf{Grafico \thefigure}#2#3}
\DeclareCaptionFormat{esempio}{\textbf{Esempio \thefigure}#2#3}
\usepackage{imakeidx}
\makeindex[options=-s ../Mod_base/oldclaudio.sti]
\input{../Mod_base/pagina}
\input{../Mod_base/indice}
\input{../Mod_base/date}
\input{../Mod_base/loghi}
\input{../Mod_base/unita_misura}
\input{../Mod_base/utili}
\input{../Mod_base/stand_class}
\usepackage{qrcode}
\newcommand{\HRule}{\rule{\linewidth}{0.5mm}}
\usepackage{placeins} 
\makeatletter
\renewcommand\frontmatter{%
	\cleardoublepage
	\@mainmatterfalse
	\pagenumbering{arabic}}
\renewcommand\mainmatter{%
	\cleardoublepage
	\@mainmattertrue}
\makeatother

%%%%%%%%%%%%%%%%%%%%%%%%%%%%%%%%
%%%lunghezza arrotondamenti%%%%%
\newcommand{\lungarrotandamento}{4}
%%%%%%%%%%%%%%%%%%%%%%%%%%%%%%%%%%
%secondo/scomposizionipoli,
  secondo/equazioni1,
% secondo/equazioni_fraz,
  %quarto/disequazioni_primogrado,
 secondo/sistemi_lineari,
	secondo/semplificazioni,
	secondo/radicali,
	secondo/Equazioni2grad,
	
\usepackage[grumpy,mark,markifdirty,raisemark=0.95\paperheight]{gitinfo2}
\usepackage[toc,page]{appendix}

\renewcommand{\appendixtocname}{Appendice}

\renewcommand{\appendixpagename}{Appendice}
\usepackage{tkz-berge}
\usepackage[italian]{varioref}
\usepackage{hyperxmp}
\usepackage[pdfpagelabels]{hyperref}
\usepackage[italian]{cleveref}
\input{../Mod_base/tcolorboxgest}
\title{Codici Calcolatrici}
\author{Claudio Duchi}
\date{\datetime}
\hypersetup{%
	pdfencoding=auto,
	urlcolor={blue},
	pdftitle={Codici Calcolatrici},
	pdfsubject={Calcolatrici},
	pdfstartview={FitH},
	pdfpagemode={UseOutlines},
	pdflicenseurl={http://creativecommons.org/licenses/by-nc-nd/3.0/},
	pdflang={it},
	pdfmetalang={it},
	pdfkeywords={goniometria, trigoniometria, numeri complessi},
	pdfcopyright={Copyright (C) 2018, Claudio Duchi},
	pdfcontacturl={http://breviariomatematico.altervista.org},
	pdfcontactpostcode={},
	pdfcontactphone={},
	pdfcontactemail={claduc},
	pdfcontactcountry={Italy},
	pdfcontactcity={Perugia},
	pdfcontactaddress={},
	pdfcaptionwriter={Claudio Duchi},
	pdfauthortitle={},%
	pdfauthor={Claudio Duchi},
	linkcolor={blue},
	colorlinks=true,
	citecolor={red},
	breaklinks,
	bookmarksopen,
	verbose,
	baseurl={http://breviariomatematico.altervista.org}
}
\listfiles
\begin{document}
\frontmatter
		\begin{titlepage}
	\begin{center}
		\input{../Mod_base/Lgrande}\\[1cm]
		\textsc{\LARGE Claudio Duchi}\\[1.5cm]
		\HRule \\[0.4cm]
		{ \huge \bfseries ESERCIZI SVOLTI DI MATEMATICA}\\[0.4cm]
		{\LARGE \textsc{TERZO MANUTENZIONE}}
		\HRule \\[1.5cm]
		\vfill
		\begin{tikzpicture}
		\renewcommand*{\VertexBallColor}{green!50!black}
		\GraphInit[vstyle=Art]
		\grComplete[RA=5]{18}
		\end{tikzpicture}
	\end{center}
	{\centering
	Release:\gitReln\ (\gitAbbrevHash)\ Autore:\gitAuthorName\ 
	\gitCommitterDate \\
}
\end{titlepage}
\setcounter{page}{2} 
\input{../Mod_base/copyright}
\tableofcontents 
%\addcontentsline{toc}{chapter}{\listtablename}
%\listoftables

\addcontentsline{toc}{chapter}{\listfigurename}
\listoffigures
\renewcommand\lstlistlistingname{Esempi e contro esempi}
\addcontentsline{toc}{chapter}{\lstlistlistingname}
\addcontentsline{toc}{section}{Esempi}
\lstlistoflistings%{}
{
%	https://tex.stackexchange.com/questions/318486/number-freestyle-causes-an-overlay-in-the-list-of-tcolorboxes/318512#318512
	\makeatletter
	\renewcommand{\l@tcolorbox}{\@dottedtocline{1}{0pt}{3em}}
	\makeatother
\tcblistof[\section*]{thm}{Esempi}
\addcontentsline{toc}{section}{Contro esempi}
\tcblistof[\section*]{cthm}{Contro esempi}
}
\mainmatter%
\include{scomposizionipoli}

   \chapter{Equazioni}
\label{sec:equazioni}
\section{Definizioni}
\begin{definizionet}{}{}
Un'equazione\index{Equazione} è l'uguaglianza fra due espressioni.
\end{definizionet}
 Dato che dipende dai valori che vengono assegnati alle lettere un'equazione è un'uguaglianza condizionata.
\begin{definizionet}{}{}
  I valori, che sostituiti alle lettere rendono vere l'uguaglianza,  sono chiamate soluzioni\index{Equazione!soluzione}
\end{definizionet}
\begin{esempiot}{}{}
Classificare le seguenti uguaglianze
\end{esempiot}
\begin{itemize}
\item $2+3=3+5$ non è un'equazione. Mancano le lettere.
\item $2+x=3x+5y$ è un'equazione. 
\item $z+3x=0$ è un'equazione.
\end{itemize}
\begin{esempiot}{}{}
Verificare se $x=7$ è soluzione di
 l'equazione $2x-5=x+2$ 
\end{esempiot} 
 $x=7$ è soluzione infatti $2\cdot 7-5=7+2$ $9=9$ l'uguaglianza è verificata. Mentre per $x=3$ $2\cdot 3-5=7+2$ $1\neq5$ quindi non è soluzione.

Le lettere sono chiamate variabili\index{Equazione!variabile} o costanti\index{Equazione!costante}. Una variabile o incognita è una quantità non nota a priori che può assumere qualunque valore. Una costante è una quantità non nota ma fissa. Di solito si usano $x,y,z$ per indicare le incognite $a,b,c,\dots$ per indicare le costanti.  

Un'equazione in cui compare una sola lettera è detta in una incognita, con due diverse in due incognite eccetera. Il segno di uguaglianza divide l'equazione in due parti, la parte a sinistra chiamata primo membro, la parte a destra secondo membro.
\section{Principi di equivalenza}
\begin{definizionet}{}{}
Due o più equazioni sono equivalenti\index{Equazione!equivalente} se hanno le stesse soluzioni.
\end{definizionet}
\begin{esempiot}{}{}
Verificare se le due equazioni sono equivalenti 
\begin{align*}
5x+4&=4x+3\\
3x+2&=2x+3
\end{align*}
\end{esempiot}
le equazioni 
hanno la stessa soluzione
\begin{align*}
5x+4&=4x+3\\
5\cdot(-1)+4&=4\cdot(-1)+3\\
-1&=-1\\
3x+2&=2x+3\\
3\cdot(-1)+2&=2\cdot(-1)+3\\
-1&=-1
\end{align*}
Quindi le due equazioni sono equivalenti.

\subsection{Primo principio di equivalenza}
\label{sec:PrimoprincipioEquivalenza}
\begin{principiot}{Primo principio}{}
Se aggiungiamo o togliamo la stessa quantità\footnote{Quantità definita} al primo e al secondo membro di una equazione,  otteniamo un'equazione  equivalente\index{Equazione!equivalente} a quella di partenza.
\end{principiot}
\begin{esempiot}{Primo principio equivalenza}{}
$8x+14=6x+10$
\end{esempiot}
\begin{align*}
8x+14&=6x+10
\intertext{aggiungendo $+10$ ad entrambi i membri}
8x+14+10&=6x+10+10\\
8x+24&=6x+20
\intertext{le due equazioni sono equivalenti infatti}
8\cdot(-2)+14&=6\cdot(-2)+10\\
8&=8\\
8\cdot(-2)+14&=6\cdot(-2)+10\\
-2&=-2
\intertext{quindi $-2$ è soluzione per entrambe}
\end{align*}
\subsection{Conseguenze primo principio}
Se un termine passa  dal primo al secondo membro di una equazione o viceversa cambia di segno\index{Regola!del trasporto}.
\begin{esempiot}{Conseguenze primo principio}{}
$ x+5 = 8$
\end{esempiot}
\begin{NodesList}
	\tikzset{LabelStyle/.style = {left=0.1cm,pos=0.5,text=red,fill=white}}
 \[ % formula no "inline"
 \begin{aligned}
 x+5 &= 8 \AddNode\\
 x +5-5 &= 8-5 \AddNode\\
 x + 0 &= 8-5 \AddNode
 \end{aligned}
 \]
 \LinkNodes{\begin{minipage}{2cm}
aggiungo $-5$ ad entrambi i membri
 \end{minipage}
 }
 \LinkNodes{ $5-5=0$ }
 \end{NodesList}
 in pratica
 \begin{NodesList}[dy=5pt,margin=3cm]
  \[ % formula no "inline"
  \begin{aligned}
  x+5 &= 8 \AddNode\\
  x  &= 8-5 \AddNode
  \end{aligned}
  \]
  \LinkNodes{\begin{minipage}{2cm}
 sposto e cambio di segno
  \end{minipage}
  }
  \end{NodesList}
Se la stessa quantità è presente nel primo o secondo membro dell'equazione allora può essere eliminata\index{Regola!cancellazione}.
\begin{esempiot}{Regola di cancellazione}{}
$ x+5 = 8+5$
\end{esempiot}
\begin{NodesList}[dy=5pt,margin=3cm]
 \[ % formula no "inline"
 \begin{aligned}
 x+5 &= 8+5 \AddNode\\
 x +5-5 &= 8+5-5 \AddNode\\
 x + 0 &= 8+0 \AddNode
 \end{aligned}
 \]
 \LinkNodes{\begin{minipage}{2cm}
aggiungo $-5$ ad entrambi i membri
 \end{minipage}
 }
 \LinkNodes{ $5-5=0$ }
 \end{NodesList}
 in pratica
 \begin{NodesList}[dy=5pt,margin=3cm]
  \[ % formula no "inline"
  \begin{aligned}
  x+5 &= 8+5 \AddNode\\
  x  &= 8 \AddNode
  \end{aligned}
  \]
  \LinkNodes{\begin{minipage}{2cm}
semplifico
  \end{minipage}
  }
  \end{NodesList}

\subsection{Secondo principio di equivalenza}
\label{sec:SecondoprincipioEquivalenza}
\begin{principiot}{Secondo principio}{}
Se moltiplichiamo o dividiamo per  la stessa quantità diversa da zero il primo e il secondo membro di una equazione,  otteniamo un'equazione  equivalente\index{Equazione!equivalente} a quella di partenza.
\end{principiot}
\begin{esempiot}{Secondo principio di equivalenza}{}
$2x+2=x+5$
\end{esempiot}
\begin{align*}
2x+2&=x+5
\intertext{moltiplico per  $+5$  entrambi i membri}
5\cdot(2x+2)&=5\cdot(x+5)\\
10x+10&=5x+25
\intertext{le due sono equivalenti infatti}
2\cdot(3)+2&=3+5\\
8&=8\\
10\cdot(3)+10&=5\cdot(3)+25\\
40&=40
\intertext{quindi $-2$ è soluzione per entrambe}
\end{align*}
\section{Forma normale}
\label{sec:formanormale}
\begin{definizionet} {}{}
Dopo aver trasportato a primo membro tutti i termini di una equazione si ottiene un polinomio ordinato e l'equazione diventa \[P(x)=0\]
In questo caso l'equazione\index{Equazione!forma normale} si dice in forma normale.
\end{definizionet}
\begin{esempiot}{Forma normale}{}
L'equazione\[3x^2+3=0\] è in forma normale.

L'equazione\[(x-2)(3x+2)+5x=0\] non è in forma normale.
\end{esempiot}
Il grado più grande dell'equazione rispetto all'incognita è detto grado dell'equazione\index{Equazione!grado}.
\section{Equazioni di primo grado}
\label{sec:equazionidiprimogrado}
\begin{definizionet}{}{}
Una equazione\index{Equazione!di primo grado} di primo grado è un'equazione della forma \[ax=b\]
$a,b\in\R$
\end{definizionet}
\begin{esempiot}{}{}
Risolvere l'equazione $8x+2(x+11) = 6x+3(x-3) $
\end{esempiot}
 \begin{NodesList}[margin=3cm]
  \begin{align*}
  8x+2(x+11) = 6x+3(x-3) \AddNode\\
  8x+2x+2  = 6x+3x-9 \AddNode\\
  \intertext{\hfil isolati i termini con l'incognita \hfil}
  8x+2x-6x-3x  = -2-9 \AddNode\\
    x  = -11 \AddNode
  \end{align*}
  \LinkNodes{Moltiplico}
  \LinkNodes{ }
  \LinkNodes{ Ottengo la soluzione}
  \end{NodesList}
  \[x=-11\]
  è soluzione
Nell'esempio che segue l'equazione è scritta in maniera più complessa. Questo impone delle priorità nella risoluzione della stessa. 
\begin{esempiot}{}{}
Risolvere l'equazione \[2(x+\dfrac{4}{3})-\dfrac{5x-3}{2}=2x+3(x+2) \]
\end{esempiot}
 \begin{NodesList}[margin=3cm]
  \begin{align*}
  2\overbrace{(x+\dfrac{4}{3})}-\dfrac{5x-3}{2}=2x+\overbrace{3(x+2)} \AddNode\\
%  \intertext{\hfil isolati i termini con l'incognita \hfil}
\overbrace{2(\dfrac{3x+4}{3})}-\dfrac{5x-3}{2}=\overbrace{2x+3x}+6 \AddNode\\
  \dfrac{6x+8}{3}-\dfrac{5x-3}{2}=5x+6   \AddNode\\
 \dfrac{12x+16-15+9=30x+31}{6}   \AddNode\\
 -3x+25=30x+31\AddNode\\
  \intertext{\hfil isolati i termini con l'incognita \hfil}
 \overbrace{-3x-30x}=\overbrace{-25+31}\AddNode\\
 -33x=6\AddNode\\
 x=-\dfrac{6}{33}\AddNode\\
 x=-\dfrac{2}{11}\AddNode\\
  \end{align*}
  \LinkNodes{Precedenze}
   \LinkNodes{Moltiplico e sommo}
  \LinkNodes{mcm}
  \LinkNodes{Moltiplico per $6$}
  \LinkNodes{Sommo}
  \LinkNodes{Separo}
  \LinkNodes{Divido}
    \LinkNodes{Semplifico}
  \end{NodesList}
Le due equazioni precedenti, in origine, non erano in forma normale\index{Equazione!forma normale}, semplificando e separando le variabili otteniamo un'equazione in forma normale,  che viene risolta dividendo per il termine davanti l'incognita. Dato che abbiamo ottenuto una soluzione l'equazione è determinata\index{Equazione!determinata}.
 
\begin{esempiot}{}{}
Risolvere l'equazione \[2(3x+2)=3(\dfrac{4}{3}x-1)+2(x+1) \]
\end{esempiot}
 \begin{NodesList}[margin=3cm]
  \begin{align*}
\overbrace{2(3x+2)}=3(\dfrac{4}{3}x-1)+\overbrace{2(x+1)} \AddNode\\
%  \intertext{\hfil isolati i termini con l'incognita \hfil}
6x+3=\overbrace{3(\dfrac{4x-3}{3})}+2x+2 \AddNode\\
 6x+3=4x-3+2x+2  \AddNode\\
  \intertext{\hfil isolati i termini con l'incognita \hfil}
 \overbrace{6x-4x-2x}=\overbrace{-4-3+2}  \AddNode\\
 0=-5\AddNode
  \end{align*}
  \LinkNodes{Precedenze}
   \LinkNodes{Moltiplico}
  \LinkNodes{Separo}
  \LinkNodes{Sommo}
  \end{NodesList}
  a  primo membro abbiamo zero al secondo meno cinque. L'uguaglianza è impossibile l'equazione è impossibile.
\begin{figure}
	\centering
	\includestandalone[width=.3\linewidth]{secondo/diagrammi/AlberoBinario1}
	\caption[]{Classificazione equazioni}
	\label{fig:AlberoBinarioeqa1}
\end{figure}

Anche ne caso che segue l'incognita scompare solo che cambia il tipo della soluzione. Nell'esempio che segue scompare l'incognita ma l'uguaglianza ottenuta non è sempre falsa ma sempre vera. L'uguaglianza è un'identità.
\begin{esempiot}{}{}
Risolviamo l'equazione \[ 6(x-3) = 3(x-1)+5(x+\dfrac{2}{5})-(2x+17)\]
\end{esempiot}
\begin{NodesList}[margin=3cm]
  \begin{align*}
\overbrace{6(x-3)} = \overbrace{3(x-1)}+5(\overbrace{x+\dfrac{2}{5}})\overbrace{-(2x+17)} \AddNode\\
%  \intertext{\hfil isolati i termini con l'incognita \hfil}
6x-18 = 3x-3+\overbrace{5(\dfrac{5x+2}{5})}-2x-17\AddNode\\
6x-18 = 3x-3+5x+2-2x-17  \AddNode\\
  \intertext{\hfil isolati i termini con l'incognita \hfil}
 \overbrace{6x-3x-5x+2x}=\overbrace{18-3+2-17}  \AddNode\\
 0=0\AddNode
  \end{align*}
  \LinkNodes{Precedenze}
   \LinkNodes{Moltiplico}
  \LinkNodes{Separo}
  \LinkNodes{Sommo}
  \end{NodesList}
  Il primo membro è uguale al secondo l'uguaglianza è sempre vera. 







%\begin{table}[H]
%\centering
%\begin{tabular}{LCR}
%\toprule
%+a&=&\ldots\\
%\ldots&=&-a\\
%\bottomrule
%\end{tabular}
%\caption{Regola del trasporto}
%\label{tab:regtrasporto}
%\end{table}
%\begin{table}[H]
%\centering
%\begin{tabular}{LCR}
%\toprule
%\dfrac{\cdots\cdots}{a}&=&\dfrac{\cdots\cdots}{a}\\
%&\\
%a\cdot\dfrac{\cdots\cdots}{a}&=&a\cdot\dfrac{\cdots\cdots}{a}\\
%&\\
%\cdots\cdots&=&\cdots\cdots\\
%\bottomrule
%\end{tabular}
%\caption{Semplificazione denominatore}
%\label{tab:Semplificazionedenominatore}
%\end{table}
%\begin{table}[H]
%
%\centering
%\begin{tabular}{CCCCL}
%\toprule
%\multicolumn{5}{c}{ax=b}\\
%\hline
%%&\\
%\multicolumn{2}{c}{coefficienti}&&soluzione&tipo soluzione\\
%\midrule
%a\neq0&b\neq0&ax=b&x=\dfrac{b}{a}&determinata\\
%%&\\
%a\neq0&b=0&ax=0&x=0&determinata\\
%%&\\
%a=0&b=0&0x=0&&indeterminata\\
%%&\\
%a=0&b\neq0&0x=b&&impossibile\\
%\bottomrule	
%\end{tabular}
%\caption{Soluzioni equazioni primo grado intere}
%\label{tab:equazioniprimogrado}
%\end{table}
%\begin{table}%
%
%\centering
%\begin{tabular}{LR}
%\toprule
%Tipo&Nome\\
%\midrule
%ax^2+c=0&Pura\\
%\hline
%\multicolumn{2}{c}{Risoluzione}\\
%\multicolumn{2}{C}{ax^2=-c}\\
%\multicolumn{2}{C}{x^2=-\dfrac{c}{a}}\\
%\multirow{3}*{Se $-\dfrac{c}{a}>0$ esistono soluzioni reali} &x_1=-\sqrt{-\dfrac{c}{a}}\\
%&\\
%&x_2=+\sqrt{-\dfrac{c}{a}}\\
%&\\
%Se -\dfrac{c}{a}<0\text{ non esistono soluzioni reali}&\\
%&\\
%\bottomrule	
%%\end{tabular}
%%\caption{Equazione secondo grado pura}
%%\label{tab:equazione2GradoPura}
%%\end{table}
%%\begin{table}%
%%
%%\centering
%%\begin{tabular}{LR}
%\toprule
%Tipo&Nome\\
%\midrule
%ax^2+bx=0&Spuria\\
%\hline
%\multicolumn{2}{c}{Risoluzione}\\
%\multicolumn{2}{C}{ax^2+bx=0}\\
%\multicolumn{2}{C}{x(ax+b)=0}\\
%\multicolumn{2}{C}{x_1=0}\\
%\multicolumn{2}{C}{ax+b=0}\\
%\multicolumn{2}{C}{x_2=-\dfrac{b}{a}}\\
%\bottomrule	
%%\end{tabular}
%%\caption{Equazione secondo grado spuria}
%%\label{tab:equazione2GradoSpuria}
%%\end{table}
%%\begin{table}%
%%
%%\centering
%%\begin{tabular}{LR}
%\toprule
%Tipo&Nome\\
%\midrule
%ax^2=0&Monomia\\
%\hline
%\multicolumn{2}{c}{Risoluzione}\\
%\multicolumn{2}{C}{ax^2=0}\\
%\multicolumn{2}{C}{x_1=0}\\
%\multicolumn{2}{C}{x_2=0}\\
%\bottomrule	
%%\end{tabular}
%%\caption{Equazione secondo grado monomia}
%%\label{tab:equazione2GradoMonomia}
%%\end{table}
%%\begin{table}%
%%
%%\centering
%%\begin{tabular}{LR}
%\toprule
%Tipo&Nome\\%
%\midrule
%ax^2+bx+c=0&Completa\\%
%\hline
%\multicolumn{2}{c}{Risoluzione}\\%
%\multirow{3}*{$b^2-4ac>0$}&x_1=\dfrac{-b+\sqrt{b^2-4ac}}{2a}\\%
%&\\
%&x_2=\dfrac{-b-\sqrt{b^2-4ac}}{2a}\\%
%\hline
%\multirow{3}*{$b^2-4ac=0$}&x_1=-\dfrac{b}{2a}\\%
%&\\
%&x_2=-\dfrac{b}{2a}\\%
%\hline
%\multirow{3}*{$b^2-4ac<0$}&\\
%&\text{nessuna soluzione reale}\\%
%&\\
%\bottomrule	
%\end{tabular}
%\caption{Equazioni secondo grado}
%\label{tab:equazione2Gradoelenco}
%\end{table}

 \chapter{Equazioni frazionarie di primo grado}
\label{cha:Equazionefrazionariaprimogrado}
\section{Definizioni}
\label{sec:definizioni}
\begin{definizionet}{}{}
Un'equazione è frazionaria\index{Equazione!frazionaria} se l'incognita compare al denominatore.
\end{definizionet}
\begin{esempiot}{}{}
Classificare le seguenti equazioni
\end{esempiot}
L'equazione seguente non è frazionaria
\[3(x+1)+(x-2)(x-\dfrac{1}{2}=0) \]
mentre
\[\dfrac{3x+2}{4x+2}+x=0 \]
è un'equazione frazionaria l'incognita è al denominatore.

Una frazione è una divisione e in una divisione non è possibile dividere per zero. Può accadere  che vi siano valori dell'incognita che rendono il denominatore uguale a zero e di conseguente impossibile la frazione. Il dominio di una funzione o campo di esistenza è l'insieme dei valori per cui la frazione esiste.
\begin{procedurat}{}{}
\begin{enumerate}
\item Per ogni frazione che contengono le incognite discuto i denominatori.
	\begin{itemize}
	\item Pongo i denominatori uguali a zero e risolvo l'equazione che ottengo.
	\item Escludo i valori trovati negandoli $\neq$
	\end{itemize}
	\item  Scompongo il fattori primi i denominatori (attenzione alla differenza fra fattori ed addendi) es: $2x$ e  $2x+1$ sono due fattori fra loro diversi.
	\item Calcolo il mcm (Fattori comuni e non comuni, presi una sola volta con il massimo esponente)
	\item Traccio la linea di frazione 
	\item Per ogni frazione divido il minimo comune multiplo per il denominatore e il risultato della divisione lo moltiplico per il numeratore ricordando che sono obbligatorie le parentesi quando 
	\begin{itemize}
	\item Moltiplico fra loro polinomi
	\item Davanti alla linea di frazione vi è un segno meno
	\end{itemize}
	\item Ottengo un unica frazione che semplifico togliendo il denominatore
	\item Eseguo i calcoli e separo le incognite che scrivo sinistra, dai numeri che scrivo a destra, ricordando che  se un termine viene spostato rispetto all'uguale cambia di segno. Attenzione Se un numero moltiplica una lettere es $2x$ è un'incognita e andrà a sinistra ,diverso da $2$ che andra a destra.
	\item Sommo fra di loro le incognite e fra di loro i numeri.
	\item Ottengo 
	\begin{itemize}
	\item Un'equazione di primo grado che risolvo dividendo  a sinistra e a destra per il numero davanti all'incognita. Attenzione ogni numero ha un segno che non può essere trascurato.
	\begin{itemize}
	\item Controllo se i risultati ottenuti  sono accettabili confrontandoli con i valori che ho escluso eventualmente scartandoli nel caso fossero uguali.
	\end{itemize}
	\item Un'uguaglianza impossibile. Esempio $0=2$
	\item Un'identità. Esempio $2=2$
	\end{itemize}
\end{enumerate}
\end{procedurat}

 \include{quarto/disequazioni_primogrado}
\chapter{Sistemi lineari in due incognite}
\label{sec:sistemiLineariInDueIncognite}
\section{Equazioni in più variabili}
\label{sec:EquazioniInPiuVariabili}
\begin{definizionet}{Equazione in più variabili}{}\index{Equazione!più variabili!definizione}
	Un'equazione in più variabili è un'equazione in più incognite.
\end{definizionet}
\begin{definizionet}{Definizione di soluzione}{}\index{Equazione!più variabili!soluzione}
Per un'equazione in più variabili, una soluzione, è un insieme ordinato di valori che verificano l'equazione.
\end{definizionet}
\begin{esempiot}{}{}
	Risolviamo l'equazione\[ 3x+y=7\]
\end{esempiot}	
Questa equazione, in due incognite, ha per soluzione
$x=2$ e $y=1$ infatti \[3\cdot2+1=7\]
Ma anche $x=3$ e $y=-2$ infatti \[3\cdot3-2=7\] è soluzione.
In genere un'equazione in più incognite ha più di una soluzione.
\begin{figure}
	\centering
\includestandalone[width=.9\textwidth]{secondo/sistemi/mappe_concettuali_sistema_1}
	\caption{Classificazione di un sistema}
	\label{fig:ClassificazioneDiUnSistema}
\end{figure}
\section{Sistemi}
\label{sec:Sistemi}
Più  equazioni  possono avere le stesse soluzioni. Un sistema è un'insieme di due o più equazioni. Risolvere un sistema è verificare che più equazioni hanno le stesse soluzioni.
\begin{definizionet}{Definizione di sistema}{}\index{Sistema!definizione}
Un sistema è un'insieme di due o più  equazioni.
\end{definizionet}
\begin{definizionet}{Definizione di grado di sistema}{}\index{Sistema!grado}\label{def:sistemaGrado}
Il grado di un sistema, di più equazioni in tante incognite,  è il prodotto dei gradi delle equazioni del sistema ridotte in forma normale.
\end{definizionet}
\begin{definizionet}{Definizione di soluzione}{}\index{Sistema!definizione!soluzione}
Una soluzione per un sistema è un insieme ordinato di valori che sono soluzione per ogni equazione del sistema.
\end{definizionet}
\section[Classificazione rispetto alle soluzioni]{Classificazione dei sistemi rispetto alle soluzioni}
\begin{definizionet}{Definizione di equivalenza}{}\index{Sistema!definizione di equivalenza}
Due sistemi sono equivalenti quando hanno lo stesso insieme soluzione.
\end{definizionet}
\begin{definizionet}{Forma normale di un sistema lineare}{}\index{Sistema!lineare!forma normale}{}
Un sistema di due equazioni di 1 grado in due incognite x, y, a coefficienti numerici, si dice ridotto in forma normale se è del tipo
\[\left\{\begin{array}{l} {ax+by=c} \\ {a'x+b'y=c'}\end{array}\right. \]
dove a, b, a', b' si chiamano coefficienti delle incognite c, c' si chiamano termini noti
\end{definizionet}
\begin{teoremat}{Teorema fondamentale}{}\index{Sistema!lineare!teorema fondamentale}{}
Se i coefficienti del sistema lineare sono diversi da zero e non sono tra loro proporzionali $\dfrac{a}{a'} \ne \dfrac{b}{b'}$ il sistema ammette una e una sola soluzione data da
\[
\begin{cases}
	x=\dfrac{b'c-bc'}{ab'-a'b}\\
	y=\dfrac{ac'-a'c}{ab'-a'b}
\end{cases}
\]
Se, invece, sono proporzionali solo i coefficienti delle incognite cioè; $\dfrac{a'}{a} =\dfrac{b'}{b} \ne \dfrac{c'}{c} $, allora il sistema è impossibile
Se, invece, sono proporzionali i coefficienti e i termini noti delle equazioni cioè; $\dfrac{a'}{a} =\dfrac{b'}{b} =\dfrac{c'}{c} $, allora il sistema è indeterminato.
\end{teoremat}
Consideriamo il sistema canonico, \[ \left\{\begin{array}{l} {ax+by=c} \\ {a'x+b'y=c'} \end{array}\right. \]
moltiplicando la prima equazione per b' e la seconda per --b, avremo  \[\left\{\begin{array}{l} {ab'x+bb'y=b'c} \\ {-ba'x-bb'y=-bc'} \end{array}\right. \]
sommando lungo le colonne otteniamo \[\dfrac{\left\{\begin{array}{l} {ab'x+bb'y=b'c} \\ {-ba'x-bb'y=-bc'} \end{array}\right. }{\left(ab'-ba'\right)x=b'c-bc'} \]  da cui  \[x=\dfrac{b'c-bc'}{ab'-a'b} \]
moltiplicando la prima equazione per a' e la seconda per --a avremo  \[\left\{\begin{array}{l} {aa'x+ba'y=a'c} \\ {-aa'x-ab'y=-ac'} \end{array}\right. \] sommando lungo le colonne otteniamo  \[\dfrac{\left\{\begin{array}{l} {aa'x+ba'y=a'c} \\ {-aa'x-ab'y=-ac'} \end{array}\right. }{\left(a'b-ab'\right)y=a'c-ac'} \] da cui \[y=\dfrac{ac'-a'c}{ab'-a'b} \]  da cui se  \[\dfrac{a'}{a} \ne \dfrac{b'}{b} \] avremo  \[ab'-a'b\ne 0\]  quindi il sistema ha una e una sola soluzione.
Se \[\dfrac{a'}{a} =\dfrac{b'}{b} \]  avremo due casi  \[\dfrac{a'}{a} =\dfrac{b'}{b} \ne \dfrac{c'}{c} \]  e  \[\dfrac{a'}{a} =\dfrac{b'}{b} =\dfrac{c'}{c} \]
Nel primo caso il sistema è impossibile perché  $ab'-a'b=0$, $a'c-ac'\ne 0$, $b'c-bc'\ne 0$  e quindi poiché è possibile riscrivere il sistema canonico nella forma  \[\left\{\begin{array}{l} {\left(ab'-ab'\right)x=b'c-bc'} \\ {\left(ab'-ab'\right)y=ac'-a'c} \end{array}\right. \]  da cui  \[\left\{\begin{array}{l} {0=b'c-bc'} \\ {0=ac'-a'c} \end{array}\right. \] impossibile.
Nel secondo caso il sistema è indeterminato perché  $ab'-a'b=0$, $a'c-ac'=0$, $b'c-bc'=0$  e quindi giacché è possibile riscrivere il sistema canonico nella forma  \[\left\{\begin{array}{l} {\left(ab'-ab'\right)x=b'c-bc'} \\ {\left(ab'-ab'\right)y=ac'-a'c} \end{array}\right. \]  da cui \[\left\{\begin{array}{l} {0=0} \\ {0=0} \end{array}\right. \]  indeterminato
\section{Metodi di risoluzione}
\label{sec:MetodiDiRisoluzione}
\subsection{Sostituzione}
\label{sec:Sostituzione}
Un sistema lineare in forma canonica si risolve con il metodo di sostituzione\index{Sistema!metodo!sostituzione} isolando una variabile in una equazione e sostituendola nelle altre. 
\begin{esempiot}{}{}
Supponiamo di avere un sistema lineare in forma normale
\[
\begin{cases}
	x+2y=1\\
	3x-y=2
\end{cases}
\]
\end{esempiot}
risolvo la prima rispetto alla x e ottengo 
\[
\begin{cases}
	x=1-2y\\
	3x-y=2
\end{cases}
\]
sostituisco nella seconda ed ottengo
\[
\begin{cases}
	x=1-2y\\
	3(1-2y)-y=2
\end{cases}
\begin{cases}
	x=1-2y\\
	3-6y-y=2
\end{cases}
\begin{cases}
	x=1-2y\\
	-7y=2-3
\end{cases}
\begin{cases}
	x=1-2y\\
	-7y=-1
\end{cases}
\begin{cases}
	x=1-2y\\
	y=\dfrac{1}{7}
\end{cases}
\begin{cases}
	x=1-2\dfrac{1}{7}\\
	y=\dfrac{1}{7}
\end{cases}
\]
\[
\begin{cases}
	x=\dfrac{7-2}{7} \\
	y=\dfrac{1}{7}
\end{cases}
\begin{cases}
	x=\dfrac{5}{7}\\
	y=\dfrac{1}{7}
\end{cases}
\]
\subsection{Confronto}
\label{sec:Confronto}
Un sistema lineare in forma canonica si risolve con il metodo del confronto\index{Sistema!metodo!confronto} risolvendo due equazioni rispetto ad un stessa variabile e confrontando i valori attenuti.
\begin{esempiot}{}{}
supponiamo di avere un sistema lineare in forma normale
\[
\begin{cases}
	x+2y=1\\
	3x-y=2
\end{cases}
\]
\end{esempiot}
Risolvo rispetto ad x e ottengo
\[
\begin{cases}
	x=-2y+1\\
	x=\dfrac{2+y}{3}
\end{cases}
\begin{cases}
	-2y+1=\dfrac{2+y}{3}\\
		x=-2y+1
\end{cases}
\begin{cases}
	-6y+3=2+y\\
	x=-2y+1
\end{cases}
\begin{cases}
	-7y=-1\\
	x=-2y+1
\end{cases}
\begin{cases}
	y=\dfrac{1}{7}\\
	x=-2y+1
\end{cases}
\]
\[
\begin{cases}
	y=\dfrac{1}{7}\\
	x=-2\dfrac{1}{7}+1
\end{cases}
\begin{cases}
	y=\dfrac{1}{7}\\
	x=\dfrac{5}{7}
\end{cases}
\]

\subsection{Riduzione}
\label{sec:Riduzionesist}
Questo metodo è chiamato anche di somma sottrazione e consiste nel sommare o sottrarre le equazioni in modo che vengano determinate le incognite. Iniziamo con un po di vocabolario, chiamiamo riga un'equazione del sistema. Mentre la colonna è formate in verticale dalla stessa incognita nelle varie equazioni.
\begin{esempiot} {}{}
	Risolvere il sistema \[
	\begin{cases}
	2y+3x=2\\
	x+y=5
	\end{cases}\]
\end{esempiot}
Il sistema è formato da due equazioni, quindi da due righe. Il sistema non è ordinato per colonne quindi bisogna riscriverlo in questo modo \[
\begin{cases}
3x+2y=2\\
x+y=5
\end{cases}\]
Il metodo consiste nel moltiplicare le righe per dei valori  opportuni e  sommare o sottrarre lungo le colonne in modo che una variabile scompaia.
\[
\begin{cases}
3x+2y=2\\
\tikzmark{1}3x+3y=15\tikzmark{2}
\end{cases}\]
\begin{tikzpicture}[remember picture, overlay]
\node[below=0.1cm and 0cm of 1](3){};
\node[below=0.1cm and 0cm of 2](4){};
\node[left=2cm and 0cm of 1](5){$3$};
\node[below right=0.2cm and 0.1cm of 5](6) {$0-y=-13$};
\draw (3) edge  (4);
\end{tikzpicture}\\

Moltiplicando la seconda riga per tre e sottraendola alla prima otteniamo che $y=13$. 

\[
\begin{cases}
3x+2y=2\\
\tikzmark{1}2x+2y=10\tikzmark{2}
\end{cases}\]
\begin{tikzpicture}[remember picture, overlay]
\node[below=0.1cm and 0cm of 1](3){};
\node[below=0.1cm and 0cm of 2](4){};
\node[left=2cm and 0cm of 1](5){$2$};
\node[below right=0.2cm and 0.1cm of 5](6) {$x+0 =-8$};
\draw (3) edge  (4);
\end{tikzpicture}\\

Ripartendo dal sistema iniziale e moltiplicando la seconda riga per due otteniamo che $x=-8$

     
	\chapter{Formule inverse}
\label{cha:semplificazioni}
\section[Primo caso]{Primo caso $a\cdot c=b$}
\label{sec:primocasosemp}
Problema: dato $a\cdot c=b$ trovare $c$.
\begin{align*}
a\cdot c=b&&\\
\text{Divido per $a$ entrambi i lati dell'uguaglianza, e ottengo}\\
\dfrac{a\cdot c}{a}=\dfrac{b}{a}&&\\
\text{Semplifico a sinistra e ottengo c}\\
c=\dfrac{b}{a}&&
\end{align*}
\section[Secondo caso]{Secondo caso $a=\dfrac{b}{c}$}
\label{sec:secondocasosemp}
Problema: dato $a=\dfrac{b}{c}$ trovare $c$.
\begin{align*}
a=\dfrac{b}{c}&&\\
\text{Moltiplico per $c$ entrambi i lati dell'uguaglianza ed ottengo }\\
a\cdot c=\dfrac{b}{c}\cdot c&&\\
\text{Semplifico a destra e ottengo}\\
a\cdot c=b&&\\
\text{Quindi procedo come in\nobs\ref{sec:primocasosemp} }
\end{align*}
\section[Terzo caso]{Terzo caso $a=\dfrac{b}{c+d}$}
\label{sec:terzocasosemp}
\subsection{Trovare b}
\label{sec:terzocasosemptrovareb}
Problema: dato $a=\dfrac{b}{c+d}$ trovare $b$
\begin{align*}
a=\dfrac{b}{c+d}&&\\
\text{Moltiplico per $c+d$ entrambi i lati dell'uguaglianza ed ottengo }\\
a\cdot (c+d)=\dfrac{b}{c+d}\cdot(c+d)&&\\
\text{Semplifico a destra e ottengo}\\
a\cdot (c+d)=b&&
\end{align*}
\subsection{Trovare d}
\label{sec:terzocasosemptrovared}
Problema: dato $a=\dfrac{b}{c+d}$ trovare $d$
\begin{align*}
a=\dfrac{b}{c+d}&&\\
\text{Moltiplico per $c+d$ entrambi i lati dell'uguaglianza ed ottengo }\\
a\cdot (c+d)=\dfrac{b}{c+d}\cdot(c+d)&&\\
\text{Semplifico a destra e ottengo}\\
a\cdot (c+d)=b&&\\
ac+ad=b&&\\
ad=-ac+b&&\\
\text{Divido per a}\\
d=\dfrac{-ac+b}{a}
\end{align*}
\section[Quarto caso]{Quarto caso $\dfrac{1}{a}=\dfrac{b+ c}{b\cdot c}$}
\label{sec:quartocasosemp}
\subsection{Trovare b}
\label{sec:quartocasosemptrovareb}
Problema: dato $\dfrac{1}{a}=\dfrac{b+ c}{b\cdot c}$
\begin{align*}
\dfrac{1}{a}=\dfrac{b+ c}{b\cdot c}&&\\
\text{calcolo il m.c.m }\\
\dfrac{b\cdot c=a\cdot(b+c)}{a\cdot b\cdot c}&&\\
\text{tolgo il m.c.m }\\
b\cdot c=a\cdot(b+c)&&\\
b\cdot c=a\cdot b+a\cdot c &&\\
b\cdot c-a\cdot b=a\cdot c &&\\
b\cdot (c-a)=a\cdot c &&\\
b=\dfrac{a\cdot c}{c-a}
\end{align*}

	\chapter{Radicali}
\label{Radicaliradici}
\section{Glossario}
\begin{table}[H]
\centering
$\sqrt[n]{a^m}=b$
\begin{itemize}
\item n Indice del radicale
\item $a^m$ Radicando
\item m Esponente o potenza del radicando
\item b Radice
\end{itemize}
\caption{Glossario}
\label{tab:RadicaliGlossario}
\end{table}
\begin{table}[H]
\centering
$
\begin{array}{rccc}
\toprule
 &\text{Indice} & \text{Potenza} & \text{Radicando} \\ 
 \midrule
 \sqrt[3]{a}& 3 &1  & a \\ 
 \sqrt[4]{a^3b}& 4 &1  & a^3b \\ 
 \sqrt[4]{a^2}& 4 &2  & a^2 \\
\sqrt{a^5}& 2 &5 & a^5 \\ 
\sqrt[2]{a^5}& 2 &5 & a^5 \\ 
\sqrt[4]{\dfrac{\left( a+b\right)^2 }{c}}& 4 &1 &\dfrac{\left( a+b\right)^2 }{c} \\
\sqrt[3]{a^2+b}& 3 &1 & a^2+b\\
\bottomrule	
\end{array}
$ 
\label{tab:esempiglossario}
\caption{Esempi Glossario}
\end{table}

\begin{table}[H]
\centering
\begin{itemize}
	\item $\sqrt[1]{a}=a\; \forall a\in\R^{+}\forall\; n\in\Ni$
	\item $\sqrt[n]{a^n}=a\; \forall a\in\R^{+}\forall\; n\in\Ni$
	\item $\left(\sqrt[n]{a}\right)^n=a\; \forall a\in\R^{+}\forall\; n\in\Ni$
	\item $\sqrt[nk]{a^{mk}}=\sqrt[n]{a^m}\;  \forall a\in\R^{+}\forall\; n,m,k\in\Ni$\label{Rad:invariantiva}\index{Radicali!proprietà invariantiva}
\end{itemize}
\label{tab:propRadicli}
\caption{Proprietà dei radicali}
\end{table}
\section{Proprietà invariantiva}
\label{Sec:Propinvariantivaradicali}
\begin{definizionet}
Si ottiene una radice equivalente moltiplicando o dividendo la potenza e l'indice per lo stesso valore diverso da zero.
\end{definizionet}
\subsection{Riduzione allo stesso indice}
\label{sec:RiduzioneAlloStessoIndice}

La Proprietà invariantiva permette di ridurre due radici allo stesso indice. Procediamo in questo modo
\begin{enumerate}
	\item Calcolo in m.c.m fra gli indici di tutte le radici
	\item Scrivo delle nuove radici di indice uguale al m.c.m. e per ogni radice
	\begin{itemize}
	\item Divido il m.c.m per l'indice  della radice e moltiplico il numero ottenuto per l'esponente m del radicando della  radice.
	\end{itemize}
\end{enumerate}

\begin{table}[H]
\centering
$
\begin{array}{cccl}
\toprule
\text{Passo} &  &  & \text{Note} \\  
\midrule
0 & \sqrt[3]{a} &\sqrt[5]{b}  &\text{Inizio} \\ 
1 & 15\div 3=5 &15\div 5=3  & mcm(3,5)=15 \\  
2 &\sqrt[15]{a^{1\cdot5}}=\sqrt[15]{a^{5}}  &\sqrt[15]{b^{1\cdot3}}=\sqrt[15]{b^{1\cdot3}}  &\text{Fine} \\
\bottomrule	
\end{array} 
$
\label{tab:Es1Ridstessoindice}
\caption{Esempio riduzione stesso indice}
\end{table}
\begin{table}[H]
\centering
$
\begin{array}{cccl}
\toprule
\text{Passo} &  &  &\text{Note} \\  
\midrule
0 & \sqrt{ab^2} &\sqrt[5]{a+b^2}  &\text{Inizio} \\ 
1 & 10\div 2=5 &10\div 5=2  & mcm(2,5)=10 \\  
2 &\sqrt[10]{a^{1\cdot 5}b^{2\cdot 5}}=\sqrt[10]{a^{2}b^{10}}& \sqrt[10]{\left( a+b^2\right)^{1\cdot 2} }=\sqrt[10]{\left( a+b^2\right)^{2} }   &\text{Fine} \\
\bottomrule	
\end{array} 
$
\label{tab:Es1Ridstessoindice2}
\caption{Esempio riduzione stesso indice}
\end{table}
\subsection{Ordinamento fra radici}
\label{sec:OrdinamentoFraRadici}
Il procedimento di riduzione allo stesso indice permette di confrontare due radici di indice diverso. Basta, dopo aver ridotto le radici allo stesso indice, confrontare i radicandi.
\begin{table}[H]
\centering
$
\begin{array}{ccccl}
\toprule
\text{Passo} &  &  &  &\text{Note} \\ 
\midrule
0 & \sqrt{5} &  &\sqrt[3]{6} & \text{Inizio} \\ 
1 & \sqrt[6]{5^3} &  &\sqrt[6]{6^2}  &\text{Riduzione stesso indice} \\ 
2 & 5^3=125 & > & 6^2=36 & \text{Confronto fra radicandi} \\ 
3 & \sqrt{5} & > & \sqrt[3]{6} & \text{Fine} \\
\bottomrule	
\end{array} 
$
\label{tab:confrontoradicali}
\caption{Esempio confronto radicali}
\end{table}

\subsection{Semplificare radici}
\label{sec:RadiciIriducibili}
La proprietà invariantiva permette di semplificare l'indice e l'esponente  di una radice.
Una radice è irriducibile se non è possibile semplificare l'indice con l'esponente\footnote{Cioè quando l'indice e l'esponente sono primi fra loro}
\begin{table}[H]
\centering
$
\begin{array}{ccl}
\toprule
\text{Passo} &  & \text{Note} \\ 
\midrule
0 &\sqrt[10]{a^2b^4} & \text{Inizio} \\ 
1 &  &  10,2,4 \text{ si dividono per }2\\ 
2 &\sqrt[5]{ab^2}  &\text{Fine}\\
\bottomrule
\end{array} 
$
\label{Tab:radiceriducibile}
\caption{Esempio radice riducibile}
\end{table}
\begin{table}[H]
\centering
$
\begin{array}{ccl}
\toprule
\text{Passo} &  & \text{Note} \\ 
\midrule
0 &\sqrt[10]{a^2+b^4} & \text{Inizio} \\ 
1 &  &  10,2,4 \text{ si dividono per }2 \text{ ma è una somma}\\ 
2 &\sqrt[10]{a^2+b^4}  &\text{Fine}\\
\bottomrule
\end{array} 
$
\label{Tab:radiceriducibilece}
\caption{Esempio radice irriducibile}
\end{table}
\begin{table}[H]
\centering
$
\begin{array}{ccl}
\toprule
\text{Passo} &  & \text{Note} \\ 
\midrule
0 &\sqrt[20]{\dfrac{a^5\left( a^5+b\right)^{10} }{x^{15}}} & \text{Inizio} \\ 
1 &  &  20,10,5,15 \text{ si dividono per }5 \\ 
2 &\sqrt[4]{\dfrac{a\left( a^5+b\right)^{2} }{x^{3}}} & \text{Fine} \\ 
\bottomrule
\end{array} 
$
\label{Tab:radiceriducibilece2}
\caption{Esempio radice riducibile}
\end{table}
\begin{table}[H]
\centering
$
\begin{array}{ccl}
\toprule
\text{Passo} &  & \text{Note} \\ 
\midrule
0 &\sqrt[10]{a^{20}}= & \text{Inizio} \\ 
1 &  &  20,10 \text{ si dividono per }10 \\ 
2 &=\sqrt[1]{a^2}=a^2 & \text{Fine} \\ 
\bottomrule
\end{array} 
$
\label{Tab:radiceriducibilece3}
\caption{Esempio radice riducibile}
\end{table}
\begin{table}[H]
\centering
$
\begin{array}{ccl}
\toprule
\text{Passo} &  & \text{Note} \\ 
\midrule
0 &\sqrt[8]{(-2)^{20}}= & \text{Inizio} \\ 
1 &  &  8,10 \text{ si dividono per }2\\
& \sqrt[4]{(-2)^{5}}&\text{ ma il radicando è negativo, quindi non esiste radice reale} \\ 
 & & \text{Fine} \\ 
\bottomrule
\end{array} 
$
\label{Tab:radiceriducibilece4}
\caption{Esempio radice non riducibile}
\end{table}
\begin{table}[H]
\centering
$
\begin{array}{ccl}
\toprule
\text{Passo} &  & \text{Note} \\ 
\midrule
0 &\sqrt[10]{(-2)^{6}}= & \text{Inizio} \\ 
1 &  &  10,6 \text{ si dividono per }2\\
& \sqrt[5]{(-2)^{3}}&\text{il radicando è negativo ma l'indice è dispari quindi esiste la radice reale} \\ 
2 &=\sqrt[5]{(-2)^{3}} & \text{Fine} \\ 
\bottomrule
\end{array} 
$
\label{Tab:radiceriducibilece5}
\caption{Esempio radice riducibile}
\end{table}
\section{Operazioni con le Radici}
\label{sec:operazioniradici}
\subsection{Prodotto}
\label{sec:prodottoradici}
Il prodotto di due o più radici, con lo stesso indice, è una radice che ha per indice lo stesso indice e per radicando il prodotto dei radicandi.

\[\sqrt[n]{a}\cdot\sqrt[n]{b}=\sqrt[n]{a\cdot b}\]
\begin{table}[H]
\centering
$
\begin{array}{ccl}
\toprule
\text{Passo} &  & \text{Note} \\ 
\midrule
0 &\sqrt[3]{a^3b}\cdot\sqrt[3]{a^4b^2x}=  & \text{Inizio} \\ 
1 &  &  \text{i due radicali hanno lo stesso indice} \\ 
2 &=\sqrt[3]{a^3b}\cdot\sqrt[3]{a^4b^2x}=\sqrt[3]{a^7b^3x} & \text{Fine} \\ 
\bottomrule
\end{array} 
$
\label{Tab:radicprodotto1}
\caption{Esempio prodotto di radici con lo stesso indice}
\end{table}
Se le radici hanno indice diverso bisogna prima ridurle allo stesso indice.
\begin{table}[H]
\centering
$
\begin{array}{ccl}
\toprule
\text{Passo} &  & \text{Note} \\ 
\midrule
0 &\sqrt[3]{a^3b}\cdot\sqrt[5]{a^4b^2x}=  & \text{Inizio} \\ 
1 &  &   \text{I due radicali hanno indice diverso} \\ 
2 &=\sqrt[15]{a^{15}b^5}\cdot\sqrt[15]{a^4b^2x}= & \text{Riduco allo stesso indice} \\ 
3 &=\sqrt[15]{a^{15}b^5}\cdot\sqrt[15]{a^4b^2x}=\sqrt[15]{a^{17}b^{11}x^3} & \text{Fine} \\ 
\bottomrule
\end{array} 
$
\label{Tab:radiceprodotto2}
\caption{Esempio prodotto di radici con indice diverso}
\end{table}
\begin{table}[H]
\centering
$
\begin{array}{ccl}
\toprule
\text{Passo} &  & \text{Note} \\ 
\midrule
0 &\sqrt[3]{\dfrac{ya}{x}}\cdot\sqrt{\dfrac{x^2}{y}}\cdot\sqrt[6]{\dfrac{y}{x^4}}  & \text{Inizio} \\ 
1 &  &   \text{I due radicali hanno indice diverso} \\ 
2 &=\sqrt[15]{a^{15}b^5}\cdot\sqrt[15]{a^4b^2x}= & \text{Riduco allo stesso indice} \\ 
3 &=\sqrt[15]{a^{15}b^5}\cdot\sqrt[15]{a^4b^2x}=\sqrt[15]{a^{17}b^{11}x^3} & \text{Fine} \\ 
\bottomrule
\end{array} 
$
\label{Tab:radiceprodotto3}
\caption{Esempio prodotto di radici con indice diverso}
\end{table}
\subsubsection{Trasporto di un termine fuori del segno di radice}
\label{sec:Trasportofuoriradici}
Se la potenza m del radicando è maggiore o uguale all'indice n del radicando allora:
\[\sqrt[n]{a^m}=a^q\sqrt[n]{a^r}\] 
\begin{itemize}
\item n Indice radice
\item m Esponente o potenza del radicando
\item q Quoziente della divisione di m per n
\item r Resto della divisione di m per n\footnote{Per ottenere r basta moltiplicare la parte decimale della divisione di per n. Es $3\div 2=1,5$ $r=0,5\cdot 2=1$ }
\end{itemize}
\begin{table}[H]
\centering
$
\begin{array}{ccc}
\toprule
\text{Passo} &  & \text{Note} \\
\midrule
0& \sqrt[3]{32}=\sqrt[3]{2^5} & \text{Inizio} \\ 
1 & \text{Divido } 5 \text{  per } 3 &q=1\text{  }r=2  \\ 
2&  \sqrt[3]{32}=\sqrt[3]{2^5}=2^1\sqrt[3]{2^2}=2\sqrt[3]{4}& \text{Fine} \\ 
\bottomrule 
\end{array}
$ 
\label{tab:Trasportofuoriradici1}
\caption{Esempio trasporto di un termine fuori del segno di radice}
\end{table}
\begin{table}[H]
\centering
$
\begin{array}{ccc}
\toprule
\text{Passo} &  & \text{Note} \\
\midrule
0& \sqrt[8]{a^{15}} & \text{Inizio} \\ 
1 & \text{Divido } 15 \text{  per } 8 &q=1\text{  }r=7  \\ 
2&  \sqrt[8]{a^{15}}=a^1\sqrt[8]{a^7}=a\sqrt[8]{a^7}& \text{Fine} \\ 
\bottomrule 
\end{array}
$ 
\label{tab:Trasportofuoriradici2}
\caption{Esempio trasporto di un termine fuori del segno di radice}
\end{table}
\begin{table}[H]
\centering
$
\begin{array}{ccc}
\toprule
\text{Passo} &  & \text{Note} \\
\midrule
0& \sqrt{72}=\sqrt{3^2\cdot 2^3} & \text{Inizio} \\ 
1 & \text{Divido } 2 \text{  per } 2 q=1\text{  }r=0 \\
&\text{ Divido } 3 \text{  per } 2\text{  } q=1\text{  }r=1    \\ 
2&  \sqrt{72}=\sqrt{3^2\cdot 2^3}=3^1\cdot 2^1\sqrt{3^0 2}=6\sqrt{2}& \text{Fine} \\ 
\bottomrule 
\end{array}
$ 
\label{tab:Trasportofuoriradici3}
\caption{Esempio trasporto di un termine fuori del segno di radice}
\end{table}
\begin{table}
\centering
$
\begin{array}{ccc}
\toprule
\text{Passo} &  & \text{Note} \\
\midrule
0& \sqrt[3]{\dfrac{a^6b^2}{c^4\left( a+b\right)^3 }} & \text{Inizio} \\ %
1 & considero gli esponenti maggiori o uguali all'indice della radice cioè 6,4,3 \\
&\text{ Divido } 6 \text{  per } 3\text{  } q=2\text{  }r=0    \\ 
&\text{ Divido } 4 \text{  per } 3\text{  } q=1\text{  }r=1    \\
&\text{ Divido } 3 \text{  per } 3\text{  } q=1\text{  }r=0    \\
2& \sqrt[3]{\dfrac{a^6b^2}{c^4\left( a+b\right)^3 }}=\dfrac{a^2}{c\left(a+b\right)}\sqrt[3]{\dfrac{a^0b^2}{c^1\left( a+b\right)^0 }} =\dfrac{a^2}{c\left(a+b\right)  }\sqrt[3]{\dfrac{b^2}{c}} & \text{Fine} \\ 
\bottomrule 
\end{array}
$ 
\label{tab:Trasportofuoriradici4}
\caption{Esempio trasporto di un termine fuori del segno di radice}
\end{table}
\subsubsection{Trasporto di un termine dentro il segno di radice}
\label{sec:Trasportodentroradici}
\[b^p\sqrt[n]{a^m}=\sqrt[n]{b^{p\cdot n}a^m}\]
\begin{table}[H]
\centering
$
\begin{array}{ccc}
\toprule
\text{Passo} &  & \text{Note} \\
\midrule
0& 3\sqrt[2]{2} & \text{Inizio} \\ 
1& 3\sqrt[2]{2}=\sqrt{3^2\cdot 2}=\sqrt[2]{18} & \text{Fine} \\ 
\bottomrule 
\end{array}
$ 
\label{tab:Trasportodentroradici1}
\caption{Esempio trasporto di un termine dentro il segno di radice}
\end{table}
\begin{table}[H]
\centering
$
\begin{array}{ccc}
\toprule
\text{Passo} &  & \text{Note} \\
\midrule
0& \dfrac{c}{\left( a+b\right)^2 }\sqrt[3]{c} & \text{Inizio} \\ 
1& \dfrac{c}{\left( a+b\right)^2 }\sqrt[3]{c}=\sqrt[3]{\dfrac{c^{1\cdot 3}c}{\left(a+b \right) }^{2\cdot 3}}=\sqrt[3]{\dfrac{c^{4}}{\left(a+b \right) }^{6}} & \text{Fine} \\ 
\bottomrule 
\end{array}
$ 
\label{tab:Trasportodentroradici2}
\caption{Esempio trasporto di un termine dentro il segno di radice}
\end{table}
\section{Potenze}
\label{sec:PotenzeRadici}
\[\left( \sqrt[n]{a}\right)^m=\sqrt[n]{a^m}\]
\begin{table}[H]
\centering
$
\begin{array}{cc}
\toprule
\left( \sqrt[4]{a}\right)^3=\sqrt[4]{a^3} \\ 
\left( \sqrt[3]{a+b}\right)^2=\sqrt[2]{\left( a+b\right)^2 } \\ 
\left( \sqrt[5]{a^2b}\right)^3=\sqrt[4]{a^6b^3} \\ 
\bottomrule 
\end{array}
$ 
\label{tab:potenzeradici1}
\caption{Esempi potenze radicali}
\end{table}
\section{Quoziente} 
\label{sec:quozienteradicali}
\[\dfrac{\sqrt[n]{a}}{\sqrt[n]{b}}=\sqrt[n]{\dfrac{a}{b}}\]
\section{Radice di radice}
\label{sec:radicediradice}
\[\sqrt[n]{\sqrt[p]{a}}=\sqrt[n\cdot p]{a}\]
\section{Somma}
\label{sec:SommaReali}
\begin{itemize}
\item Due radici sono simile se hanno lo stesso indice e  lo stesso radicando.
\item Se consideriamo una radice come un monomio avremo che il numero che compare davanti al simbolo di radice è la "`parte numerica' mentre la radice è la "`parte letterale'. Possiamo dare la seguente definizione: La somma di due radici simili è una radice simile alle precedenti che ha per parte numerica la somma algebrica delle parti numeriche.
\end{itemize}
\section{Razionalizzazione del denominatore}
\label{sec:razzionalizzazionedenominatoreradici}
\subsection{Primo caso}
\bassapriorita{Inserire razionalizzazioni}
\label{sec:razionzinalizzadioneden1caso}
\subsection{Secondo caso}
\label{sec:razionzinalizzadioneden2caso}



	\chapter{Equazioni secondo grado}
\label{cha:equazioni2grado}
\section{Equazioni di secondo grado pure}
\begin{definizionet}{Equazione di secondo grado pura}{}
	Un'equazione\index{Equazione} di secondo grado è un'equazione del tipo\[ ax^2+c=0\] $a,c\in\R\;a\neq 0$
\end{definizionet}
\begin{esempiot} {}{}
	Sono equazioni di secondo grado pure le seguenti:
      \[2x^2+3=0 \]
      \[3x^2-4=0 \]
\end{esempiot}
\section{Equazioni di secondo grado spurie}
\begin{definizionet}{Equazione di secondo grado spuria}{}
	Un'equazione\index{Equazione} di secondo grado è un'equazione del tipo\[ ax^2+bx=0\] $a,b\in\R\;a\neq 0$
\end{definizionet}
\section{Equazioni di secondo grado monomie}
\begin{definizionet}{Equazione di secondo grado monomia}{}
	Un'equazione\index{Equazione} di secondo grado è un'equazione del tipo\[ ax^2=0\] $a\in\R\;a\neq 0$
\end{definizionet}
\section{Equazioni di secondo grado complete}
\begin{definizionet}{Equazione di secondo grado completa}{}
	Un'equazione\index{Equazione} di secondo grado è un'equazione del tipo\[ ax^2+bx+c=0\] $a,b,c\in\R\;a\neq 0$
\end{definizionet}

\begin{figure}
\centering
%\includegraphics[scale=0.80]{secondo/equazioni2gradopdf-crop.pdf}
\includestandalone[width=.9\textwidth]{secondo/diagrammi/mappe_concettuali_equa_sgrado_1}
\caption{Equazioni secondo grado}
\label{fig:equazioni2gradocmap}
\end{figure}
\begin{table}%
\centering
\begin{tabular}{LR}
\toprule
Tipo&Nome\\
\midrule
ax^2+c=0&Pura\\
\hline
\multicolumn{2}{c}{Risoluzione}\\
\multicolumn{2}{C}{ax^2=-c}\\
\multicolumn{2}{C}{x^2=-\dfrac{c}{a}}\\
\multirow{3}*{Se $-\dfrac{c}{a}>0$ esistono soluzioni reali} &x_1=-\sqrt{-\dfrac{c}{a}}\\
&\\
&x_2=+\sqrt{-\dfrac{c}{a}}\\
&\\
Se -\dfrac{c}{a}<0\text{ non esistono soluzioni reali}&\\
&\\
\bottomrule	
%\end{tabular}
%\caption{Equazione secondo grado pura}
%\label{tab:equazione2GradoPura}
%\end{table}
%\begin{table}%
%
%\centering
%\begin{tabular}{LR}
\toprule
Tipo&Nome\\
\midrule
ax^2+bx=0&Spuria\\
\hline
\multicolumn{2}{c}{Risoluzione}\\
\multicolumn{2}{C}{ax^2+bx=0}\\
\multicolumn{2}{C}{x(ax+b)=0}\\
\multicolumn{2}{C}{x_1=0}\\
\multicolumn{2}{C}{ax+b=0}\\
\multicolumn{2}{C}{x_2=-\dfrac{b}{a}}\\
\bottomrule	
%\end{tabular}
%\caption{Equazione secondo grado spuria}
%\label{tab:equazione2GradoSpuria}
%\end{table}
%\begin{table}%
%
%\centering
%\begin{tabular}{LR}
\toprule
Tipo&Nome\\
\midrule
ax^2=0&Monomia\\
\hline
\multicolumn{2}{c}{Risoluzione}\\
\multicolumn{2}{C}{ax^2=0}\\
\multicolumn{2}{C}{x_1=0}\\
\multicolumn{2}{C}{x_2=0}\\
\bottomrule	
%\end{tabular}
%\caption{Equazione secondo grado monomia}
%\label{tab:equazione2GradoMonomia}
%\end{table}
%\begin{table}%
%
%\centering
%\begin{tabular}{LR}
\toprule
Tipo&Nome\\%
\midrule
ax^2+bx+c=0&Completa\\%
\hline
\multicolumn{2}{c}{Risoluzione}\\%
\multirow{3}*{$b^2-4ac>0$}&x_1=\dfrac{-b+\sqrt{b^2-4ac}}{2a}\\%
&\\
&x_2=\dfrac{-b-\sqrt{b^2-4ac}}{2a}\\%
\hline
\multirow{3}*{$b^2-4ac=0$}&x_1=-\dfrac{b}{2a}\\%
&\\
&x_2=-\dfrac{b}{2a}\\%
\hline
\multirow{3}*{$b^2-4ac<0$}&\\
&\text{nessuna soluzione reale}\\%
&\\
\bottomrule	
\end{tabular}
\caption{Equazioni secondo grado}
\label{tab:equazione2Gradoelenco}
\end{table}

	\chapter{Esempi}
	\section{Circuiti e reti}
	\label{sec:CircuitieReti}
	\begin{table}[H]
		\caption{In un circuito con due resistenze $R_1$ e $R_2$ in parallelo, trovare la formula che da $R_2$ note $R_1$ e $R_{eq}$}
		\label{tab:Trovarediffangoli}
		\begin{enumerate}
			\item Prerequisiti 
			\begin{itemize}
				\item mcm
				\item Equazioni di primo grado
				\item Resistenze in parallelo
				\item $\dfrac{1}{R_{eq}}=\dfrac{1}{R_1}+\dfrac{1}{R_2}+\cdots+\dfrac{1}{R_n}$			
			\end{itemize}
			\item Scopo: Determinare una resistenza note l'altra e la resistenza equivalente in un circuito in parallelo
			\item Testo: Determinare $R_1$ noti $R_2$ e $R_{eq}$
			\item Svolgimento: Si usa la formula che da la resistenza equivalente in parallelo.
			\begin{enumerate}
				\item $\dfrac{1}{R_{eq}}=\dfrac{1}{R_1}+\dfrac{1}{R_2}+\cdots+\dfrac{1}{R_n}$
				\item $\dfrac{1}{R_{eq}}=\dfrac{1}{R_1}+\dfrac{1}{R_2}$
				\item trovo mcm fra $R_1$, $R_2$ e $R_{eq}$
				\item $\dfrac{R_1\cdot R_2=R_{eq}\cdot (R_1+R_2)}{R_1\cdot R_2\cdot R_{eq}}$
				\item $R_1\cdot R_2=R_{eq}\cdot (R_1+R_2)$
				\item $R_1\cdot R_2=R_{eq}\cdot R_1+R_{eq}\cdot R_2$
				\item $R_1\cdot R_2-R_{eq}\cdot R_1=R_{eq}\cdot R_2$
				\item $R_1\cdot (R_2-R_{eq})=R_{eq}\cdot R_2$
				\item $R_1=\dfrac{R_{eq}\cdot R_2}{(R_2-R_{eq})}$
			\end{enumerate}
		\end{enumerate}
	\end{table}
 % % % % % % % % % % % %FINE SECONDO


\backmatter
\begin{appendices}
	\input{../Mod_base/MezziUsati}
\end{appendices}
\addcontentsline{toc}{chapter}{\indexname}
\printindex
\end{document}
